\chapter{Dynamical Systems}
\label{chap:ch3}


\section{Defining ODEs}
\begin{definition}
    \textit{An \textbf{ordinary differential equation} is an equation of an unknown function of one variable. This can be expressed as a function \textbf{OF} this unknown function and its various derivatives}.
\end{definition}
Its general form looks something like:
\begin{equation}\label{eq:3.1.1}
    F(t,y(t),y'(t),....,y^{(n)}(t))=0,
\end{equation}
Where $y(t)$ is the unknown function of independent variable $t$ and $F : \Omega \rightarrow \mathbb{R},\hspace{0.1cm}\Omega \subseteq \mathbb{R}^{n+1}$.

\begin{definition}
    \textit{The \textbf{order} of an ODE is the highest order of the derivative present in the equation.}
\end{definition}

In our case, the order is $n$.
If, however $F$ satisfies the regularity condition of the implicit theorem then the equation can be written in a much more digestible form.

\begin{theorem}
    \textit{If $F$ is continuously differentiable in $\Omega$ then the ODE can be written as:}
    \begin{equation}\label{eq:3.1.2}
        y^{(n)}=f(t,y,y',...,y^{(n-1)}),
    \end{equation}
\end{theorem}

% zi aici ce inseamna continuously differentiale sau cauta pen et naibii ce inseamna condiia de regulariate ea teoremei aleia

\section{Forming ODE systems}


\section{Applications of dynamical systems and classifications}
