\chapter{Dynamical Systems}
\label{chap:ch3}


\section{Defining Ordinary Differential Equations}
\begin{definition}
    \textit{An \textbf{ordinary differential equation} is an equation of an unknown function of one variable. This can be expressed as a function \textbf{of} this unknown function and its various derivatives}.
\end{definition}
Its general form looks something like:
\begin{equation}\label{eq:3.1.1}
    F(t,y(t),y'(t),....,y^{(n)}(t))=0,
\end{equation}
Where $y(t)$ is the unknown function of independent variable $t$ and $F : \Omega \rightarrow \mathbb{R},\hspace{0.1cm}\Omega \subseteq \mathbb{R}^{n+1}$. \\
This would be what's called the implicit form of the differential equation.

\begin{definition}
    \textit{The \textbf{order} of an ODE is the highest order of the derivative present in the equation.}
\end{definition}
In our case, the order is $n$.
If, however $F$ satisfies the regularity condition \textbf{of the implicit function theorem} then the equation can be written in a much more digestible form. \\

The \textbf{implicit function theorem} allows one to convert a relation (implicit form) to functions of some variables. These functions may not be unique, but togather, their graph may locally satisfy the relation. \\
As an example: The relation for a cardiod cannot be expressed as a sole function, we'd instead need the union of the graph of two separate functions for expressing it.
(si aici gen faci tu niste graphuri cu maple sau alte d-astea vezi cum faci virtual machineu ala sau mergi la faculta la un pc cu maple si iti faci lista cu ce vrei sa graphuiesti) \\ \\
\textbf{Reminder}: A function $f(x)$ is continuously differentiable if $\exists f'(x)$ and $f'(x) \in C^n$ where $n>=0$

\begin{theorem}
    Let $F:D \subseteq \mathbb{R}^{n+1}\rightarrow\mathbb{R}$ be a continuously differentiale function with the relation that $F(t,y(t),y'(t),....,y^{(n)}(t))=0$. Let us express a point in the set $\mathbb{R}^{n+1} =\mathbb{R}\bigtimes\mathbb{R}^n$ as (t, \textbf{Y}) = $(t, y,y', \dots, y^{ (n) })$ and fix one such point s.t. $F(t, \textbf{Y})=0$.
    So $\exists \square_{(t, \textbf{Y})} \ni U \bigtimes V \subseteq D$ s.t. $F \in C^1(U\bigtimes V)$ and $\frac{\partial F}{\partial y}(t, \textbf{Y}) \neq 0.$ Then this means $ \exists \square_{t} \ni U_0 \subseteq U,
        \square_{\textbf{Y}} \ni V_0 \subseteq V$ and a function $f : U_0 \rightarrow V_0$ s.t. $f(t) = \textbf{Y}$ and $F(t,f(t))=0, \forall t \in U_0; f \in C^1(U_0)$ and $\frac{\partial f}{\partial t_i} = - \frac{\frac{\partial F}{\partial t_i}(t,f(t))}{\frac{\partial F}{\partial y}(t,f(t))}, \forall i = \overline{1,n} , \forall t \in U_0$ and $F \in C^1(U \bigtimes V),K \in N^* \Rightarrow f \in C^K(U_0).$
\end{theorem}

So restricting the domain $\Omega$ to one that allows the implicit form to be represented as a function of the form
\begin{equation}\label{eq:3.1.2}
    y^{(n)}(t)=f(t,y(t),y'(t),...,y^{(n-1)}(t))
\end{equation}
would yield what's called the \textbf{explicit} form of the ODE. \\

A \textbf{solution} is an expression of the unknown function $y(t)$ which satisfies the relation \\
$y^{(n)}(t) = f(t,y(t),y'(t),\dots,y^{(n-1)}(t))$. A \textbf{general solution} includes all of these functions and usually has some constants of integration in the expression, while a \textbf{particular solution} doesn't have them. \\
A particular solution is usually found if we also have some initial conditions given for the unknown function, like $y(a)=b, y'(a)=c$ for some $a,b,c  \in \mathbb{R}$.
Or, more formally:

\begin{definition}
    A function $y:I \rightarrow \mathbb{R}$ is a solution of the equation \ref{eq:3.1.1} if the following conditions are met:

    1. $I \subseteq \mathbb{R}$ is nondegeneate interval. ($|I|>1$)

    2. $y \in C^n(I)$ and $(t,y(t), y'(t), \dots, y^{(n)}(t)) \in D_f, \forall t \in I$.

    3. $y^{(n)}(t)= f(t,y(t),y'(t),\dots,y^{ (n-1) }(t)), \forall t \in I$. ( \ref{eq:3.1.2} is satisfied)

\end{definition}

\section{Forming Ordinary Differential Equation systems}
We may take a sequence of such oradinary differential equationsin order to form a system:

$$
    \begin{cases}
        y'_1(t) = f_1(t,y_1(t),y_2(t),\dots,y_n(t)) \\
        y'_2(t) = f_2(t,y_1(t),y_2(t),\dots,y_n(t)) \\
        \vdots                                      \\
        y'_n(t) = f_n(t,y_1(t),y_2(t),\dots,y_n(t))
    \end{cases}
$$



\section{Applications of dynamical systems and classifications}
