\chapter{Dynamical Systems}
\label{chap:ch3}


\section{Defining ODEs}
\begin{definition}
    \textit{An \textbf{ordinary differential equation} is an equation of an unknown function of one variable. This can be expressed as a function \textbf{of} this unknown function and its various derivatives}.
\end{definition}
Its general form looks something like:
\begin{equation}\label{eq:3.1.1}
    F(t,y(t),y'(t),....,y^{(n)}(t))=0,
\end{equation}
Where $y(t)$ is the unknown function of independent variable $t$ and $F : \Omega \rightarrow \mathbb{R},\hspace{0.1cm}\Omega \subseteq \mathbb{R}^{n+1}$. \\
This would be what's called the implicit form of the differential equation.

\begin{definition}
    \textit{The \textbf{order} of an ODE is the highest order of the derivative present in the equation.}
\end{definition}
In our case, the order is $n$.
If, however $F$ satisfies the regularity condition \textbf{of the implicit function theorem} then the equation can be written in a much more digestible form. \\
Restricting the domain $\Omega$ to one that allows the implicit form to be represented as a function of the form
$$
    y^{(n)}(t)=f(t,y(t),y'(t),...,y^{(n-1)}(t))
$$
would yield what's called the \textbf{explicit} form of the ODE. \\

The \textbf{implicit function theorem} allows one to convert a relation (implicit form) to functions of some variables. These functions may not be unique, but togather, their graph may locally satisfy the relation. \\
As an example: The relation for the unit circle cannot be expressed as a sole function, we'd instead need the union of the graph of two sepperate functions for expressing it.
(si aici gen faci tu niste graphuri cu maple sau alte d-astea vezi cum faci virtual machineu ala sau mergi la faculta la un pc cu maple si iti faci lista cu ce vrei sa graphuiesti mai bine faci un cardiod si arati ce 2 functii pot sa reprezinte cardioidu ala) \\
\textbf{Reminder}: A function $f(x)$ is continuously differentiable if $\exists f'(x)$ and $f'(x) \in C^n$ where $n>=0$


\begin{theorem}
    Let $F:\mathbb{R}^{n+1}\rightarrow\mathbb{R}$ be a continuously differentiale function. Let us express a point in the set $\mathbb{R}^{n+1} =\mathbb{R}\bigtimes\mathbb{R}^n$ as (t, \textbf{y}) = $(t, y_1, \dots, y_n)$
\end{theorem}

% zi aici ce inseamna continuously differentiale sau cauta pen et naibii ce inseamna condiia de regulariate ea teoremei aleia

\section{Forming ODE systems}


\section{Applications of dynamical systems and classifications}
