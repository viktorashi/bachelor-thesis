\chapter{Introduction}

%\chapter*{Introducere}
\label{intro}

Life, is run by chemical reactions, which regulate, and propagate throughout the organism. These processes, from the beating of our hearts to the firing of our neurons, are modeled by the principles of dynamical systems. An important question is their oscillatory behavior. Can a network of chemical reactions act like a biological clock, producing sustained, periodic patterns? Or does it behave like a switch, capable of settling into multiple stable states? Here we'll discuss the former behaviour.

This thesis delves into the heart of this question by investigating the existence and absence of a specific mathematical phenomenon known as a Hopf bifurcation within a crucial class of biochemical systems: cyclic phosphorylation-dephosphorylation reaction networks. A Hopf bifurcation is a critical point at which a system's stable equilibrium gives way to sustained oscillations. Proving whether such a bifurcation can occur is paramount to determining if a given network has the intrinsic ability to generate periodic behavior.

To achieve this goal, this work embarks on a journey from fundamental principles to advanced application. We begin by establishing the mathematical language required to describe these systems, introducing the core concepts of Chemical Reaction Network (CRN) theory and the fundamentals of dynamical systems and ordinary differential equations. We will build the theoretical apparatus step-by-step, providing the reader with the geometrical intuition and analytical tools necessary to understand the conditions under which bifurcations arise.

Through researching this foundational knowleedge, I've stumbled around a fact I wasn't able to find \textbf{ proofs } on, so I tried making a short one myself: Demonstrating that the orbits of a dynamical system form a partition of its phase space using only elementary set theory instead of equivalence relations which I've seen before and frankly can't fully grasp.

We apply these tools to phosphorylation-dephosphorylation cycles and will develop and rigorous criteria, leveraging the properties of the Hurwitz matrix and convex parameters, to either rule out or confirm the possibility of Hopf bifurcations within these networks. Through detailed analysis, we will pinpoint initial conditions and parameter ranges that give rise to oscillatory dynamics.

Finally, "Open-CoNtRol" is presented. A web-based application developed to overcome the significant visualization and computational hurdles inherent in this field of study. This tool, which builds upon and contributes to established open-source scientific libraries, empowers researchers and students to numerically analyze, simulate, and visualize the complex behaviors of chemical reaction networks.

In essence,  by combining rigorous mathematical proof with the development of a practical software tool, it provides both the theoretical framework and the applied methodology for proving the existence and absence of Hopf bifurcations.

I've also had the opportunity of presenting this work at the Student Scientific Communication Session in Mathematics, on May 24th, 2025.