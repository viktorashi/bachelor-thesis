\chapter{Introduction to Chemical Reaction Networks}\label{chap:ch1}

In this first chapter we'll start with the concepts that apply to the broader range of \text{natural sciences}. We'll introduce and mathematically define

The following picture, that's also been shown by my Professor in his presentation of this topic might be a good introduction to this topic:

\hfill\break
//TODO: sa gasesti poza aia ca s-a cam luat de pe net
si sa mai scrii niste chestii din astea in mare de despre ce inseamna disciplina asta de prin paperuri sau istorie de prin ce carti gasesti sa faci o introdcuere pe bune, eu ma pun sa scriu speicfic de matele pe care il gasim
\hfill\break

A Chemical Reaction Network (CRN) consists of:

the set $S = \left\{ X_1, \dots, X_n \right\}, \left| S \right| = n$ of chemical species,

and $r$ reactions of the form:
\begin{equation}\label{mass-action_network}
	\alpha_{1j} X_1 + \dots + \alpha_{n j} X_n \xrightarrow{k_j} \beta_{1j} X_1 + \dots + \beta_{n j}, \forall j = \overline{1,r}.	
\end{equation}

Where $\alpha_{ij}$ is the stoichiometric coefficient of the reactant $X_i$ in reaction $j$.

The same goes for $\beta_{ij}$, but for the \textbf{produced} $X_i$ in reaction $j$.

$k_j$ are the reactions' rate constants.

We denote the species' concentrations with $[X_1], \dots, [X_n]$.

Now we may define the mass-action \textbf{reaction rate} of this particular reaction, which depends on the rate constant, the concentrations of reacting species and their molecularities.
\begin{equation}\label{reaction_rate}
	v = k_j [X_1]^{\alpha_{1j}} \dots [X_n]^{\alpha_{nj}}
\end{equation}

\hfill\break
//TODO poti sa mai zici si de alea de non-mass action si de unde a aparut si de unde a disparut sau cum se foloseste fiecare, si ala de mass action
\hfill\break
A pretty common example would be:
\[
	2H_2 + O_2 \xrightarrow{k} 2H_2O
\]
Where its reaction vector is:
\[
	\bordermatrix{\cr \cr
		H_2 & -2 \cr
		O_2 & -1 \cr
		H_2O & 2 \cr
	}
\]
with reaction rate $v(x) = k[H_2]^2[O_2]$. Or, by using the vector $x=(x_1, x_2, x_3, x_4)$: $v(x) = k x_1^2 x_2$.
So the reaction's dynamical system (defined more in detail in \ref{3.2.1}) is:
\begin{align}\label{1st_example_dyn_system}
	\frac{d}{dt}
	\begin{pmatrix*}
		x_1 \\
		x_2 \\
		x_3
	\end{pmatrix*} =
	v(x)
	\begin{pmatrix}
		-2 \\
		-1 \\
		2
	\end{pmatrix}
\end{align}
But if we have multiple such reactions we need to use their $\ldots$
\begin{definition}
	\textbf{Stoichiometric matrices.}
	\begin{align*}
		\Gamma_L, \Gamma_R \in \mathcal{M}_{n \bigtimes r}(\mathbb{N}) \\
		(\Gamma_L)_{ij} = \alpha_{ij} \text{ from \ref{mass-action_network}} \\
		(\Gamma_R)_{ij} = \beta{ij} \text{ from \ref{mass-action_network}}
	\end{align*}
	$\Gamma_L$ the \textbf{left} stoichiometric matrix, \\
	$\Gamma_R$ the \textbf{right} stoichiometric matrix, and \\
	$\Gamma = \Gamma_R - \Gamma_L$ : the \textbf{full} stoichiometric matrix.
\end{definition}
So constructing their equivalent of \ref{1st_example_dyn_system}:
\begin{align}
	\frac{d}{dt}
	\begin{pmatrix*}
		x_1 \\
		\vdots \\
		x_n
	\end{pmatrix*} = \Gamma
	\begin{pmatrix*}
		v_1(x)	 \\
		\vdots \\
		v_r(x)
	\end{pmatrix*}
	\text{ or, even more shorthand:  }
	\frac{dx}{dt} = \Gamma v(x).
\end{align}
Where $v(x) \in \mathcal{M}_{r \bigtimes 1}$ is the column vector of all reaction rates for the reactions.

\hfill\break
//TODO: poti sa scrii in general de ce ala un interaction graph din cartea aia de-ai gasit-o, si dupa sa zici ca poza aia de-o punem la phorolylation de-hpoysohp e un interaction graph pt networku asta: ge
\hfill\break