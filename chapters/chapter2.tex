\chapter{Introduction to Chemical Reaction Networks}\label{chap:ch1}

In this first chapter we'll start with the concepts that apply to the broader range of \text{natural sciences}. We'll introduce and mathematically define

The following picture, that's also been shown by my Professor in his presentation of this topic might be a good introduction to this topic:

\hfill\break
//TODO: sa gasesti poza aia ca s-a cam luat de pe net
si sa mai scrii niste chestii din astea in mare de despre ce inseamna disciplina asta de prin paperuri sau istorie de prin ce carti gasesti sa faci o introdcuere pe bune, eu ma pun sa scriu speicfic de matele pe care il gasim
\hfill\break

A Chemical Reaction Network (CRN) consists of:

the set $S = \left\{ X_1, \dots, X_n \right\}, \left| S \right| = n$ of chemical species,

and $r$ reactions of the form:
\begin{equation}\label{mass-action_network}
	\alpha_{1j} X_1 + \dots + \alpha_{n j} X_n \xrightarrow{k_j} \beta_{1j} X_1 + \dots + \beta_{n j}, \forall j = \overline{1,r}.	
\end{equation}

Where $\alpha_{ij}$ is the stoichiometric coefficient of the reactant $X_i$ in reaction $j$.

The same goes for $\beta_{ij}$, but for the \textbf{produced} $X_i$ in reaction $j$. These are also called the \textbf{stoichiometries} of each reactant / product.

$k_j$ are the reactions' rate constants.

We denote the species' concentrations with $[X_1], \dots, [X_n]$.

We could shorthand both of them in their vector form:
\begin{gather*}
	x \in \mathbb{R}^n_{> 0},  k \in \mathbb{R}^r_{\geq 0} \\
	x =
	\begin{pmatrix}
		x_1 \\
		\vdots \\
		x_n 	
	\end{pmatrix}, \quad k =
	\begin{pmatrix*}
		k_1  \\
		\vdots \\
		k_r	
	\end{pmatrix*}
\end{gather*}
Now we may define the mass-action \textbf{reaction rate} of this particular reaction, which depends on the rate constant, the concentrations of reacting species and their molecularities.
\begin{definition}
	\textbf{Reaction Rate}:	
	\begin{align}\label{reaction_rate}
		v : \mathbb{R}^r_{\geq 0}  &\bigtimes \mathbb{R}^n_{> 0} \rightarrow \mathbb{R}_{\geq 0}  \nonumber \\
		v(k,x) &= k_j [X_1]^{\alpha_{1j}} \dots [X_n]^{\alpha_{nj}} \\
		&= k_j x_1^{\alpha_{1j}} \dots x_n^{\alpha_{nj}} \nonumber
	\end{align}
\end{definition}

\hfill\break
//TODO poti sa mai zici si de alea de non-mass action si de unde a aparut si de unde a disparut sau cum se foloseste fiecare, si ala de mass action
\hfill\break

The way we use such a model for representing chemical reactions is with the premise that the concentrations of the species are not affected by their spatial distribution, so no funky spatial gradients and partial differential equations. Implying the "room" they have is considered necessary, the only factor in the reactions' rate being their "active mass", meaning their concentration or activity.

Otherwise, way more complex models would be used. Ones that take into account things like intermolecular interactions, the mixing process of the solution.

A pretty common example would be:
\[
	2H_2 + O_2 \xrightarrow{k} 2H_2O
\]
Where its reaction vector is:
\[
	\bordermatrix{\cr \cr
		H_2 & -2 \cr
		O_2 & -1 \cr
		H_2O & 2 \cr
	}
\]
with reaction rate $v(k,x) = k[H_2]^2[O_2]$. Or, by using the vector $x=(x_1, x_2, x_3, x_4)$: $v(x) = k x_1^2 x_2$.
So the reaction's dynamical system (defined more in detail in \ref{3.2.1}) is:
\begin{align}\label{1st_example_dyn_system}
	\frac{d}{dt}
	\begin{pmatrix*}
		x_1 \\
		x_2 \\
		x_3
	\end{pmatrix*} =
	\begin{pmatrix}
		-2 \\
		-1 \\
		2
	\end{pmatrix}
	v(k,x)
\end{align}
But if we have multiple such reactions we need to use their $\ldots$
\begin{definition}
	\textbf{Stoichiometric matrices.}
	\begin{align*}
		\Gamma_L, \Gamma_R &\in \mathcal{M}_{n \bigtimes r}(\mathbb{N}) \\
		(\Gamma_L)_{ij} &= \alpha_{ij} \text{ from \ref{mass-action_network}} \\
		(\Gamma_R)_{ij} &= \beta{ij} \text{ from \ref{mass-action_network}} \\
		\Gamma &\in \mathcal{M}_{n \bigtimes r}(\mathbb{Z}) \\
		\Gamma &= \Gamma_R - \Gamma_L
	\end{align*}
	$\Gamma_L$ the \textbf{left} stoichiometric matrix, \\
	$\Gamma_R$ the \textbf{right} stoichiometric matrix, and \\
	$\Gamma$ : the \textbf{full} stoichiometric matrix.
\end{definition}
So constructing their equivalent of \ref{1st_example_dyn_system}:
\begin{align}\label{crn_system_matrix_form}
	\frac{d}{dt}
	\begin{pmatrix*}
		x_1 \\
		\vdots \\
		x_n
	\end{pmatrix*} = \Gamma
	\begin{pmatrix*}
		v_1(k,x)	 \\
		\vdots \\
		v_r(k,x)
	\end{pmatrix*}
	\text{ or, even more shorthand:  }
	\frac{dx}{dt} = \Gamma v(k,x).
\end{align}
Where
\begin{definition}\label{flux_vector}
	\textbf{Flux Vector.}
	\begin{align}
		v : \mathbb{R}^r_{\geq 0}  \bigtimes &\mathbb{R}^n_{> 0} \rightarrow \mathbb{R}^r_{\geq 0}  \nonumber \\
		(v(x,k))_{i} &= v_i(k_i, x)
	\end{align}
	is a column vector of all reaction rates for the reactions. For reasons later discussed in \ref{convex_paramteres}, we will call it the \textbf{flux vector.}
\end{definition}

\textbf{Example}\label{bigger_network_example1} 1:

One such bigger network would be the following \textbf{open network}.
\begin{align*}
	A + C \xrightarrow{k_{1}} 2A \\
	B + C \xrightarrow{k_{2}} 2B  \\
	A \xrightarrow{k_{3}} \emptyset \\
	B \xrightarrow{k_{4}} \emptyset \\
	\emptyset \xrightarrow{k_{5}} C
\end{align*}
Or, where the last $3$ degradation / production reactions could also be written as
\begin{align*}
	A \xrightarrow{k_{3}} \\
	B \xrightarrow{k_{4}} \\
	\xrightarrow{k_{5}} C	
\end{align*}
To indicate interaction with the "outside" world.
Here: 	

$C$ is like a "resource" molecule.

$A$, $B$ \textbf{autocatalyze} themselves with $C$.

\hfill\break
//TODO: Defineste si ce e ala un open closed network, autocalalyzation etc.
\hfill\break

This system would look like:
\begin{align*}
	\Gamma_L =
	\begin{pmatrix}
		1 & 0 & 1 & 0 & 0 \\
		0 & 1 & 0 & 1 & 0 \\
		1 & 1 & 0 & 0 & 0
	\end{pmatrix}
	, \quad
	\Gamma_R =
	\begin{pmatrix}
		2 & 0 & 0 & 0 & 0 \\
		0 & 2 & 0 & 0 & 0 \\
		0 & 0 & 0 & 0 & 1
	\end{pmatrix}
\end{align*}
\begin{equation*}
 \Gamma =
 \begin{pmatrix*}
		1 & 0 & -1 & 0 & 0 \\
		0 & 1 & 0 & -1 & 0 \\
		-1 & -1 & 0 & 0 & 1
 \end{pmatrix*}
\end{equation*}
Whereas, the reaction rate:
\[
	v(k,x) =
	\begin{pmatrix*}
		v_{1}(k,x)	 \\
		v_{2}(k,x)	 \\
		v_{3}(k,x)	 \\
		v_{4}(k,x)	 \\
		v_{5}(k,x)
	\end{pmatrix*} =
	\begin{pmatrix}
		k_1 x_1 x_3 \\
		k_2 x_2 x_3 \\
		k_3 x_1 \\
		k_4 x_2 \\
		k_5
	\end{pmatrix}
\]
And so its dynamical system in matrix form is:
\begin{align*}
	\frac{dx}{dt} &= \Gamma v(k,x) \\
	\frac{d}{dt}
	\begin{pmatrix*}
		x_1	\\
		x_2	\\
		x_3	
	\end{pmatrix*} &=
	\begin{pmatrix*}
		1 & 0 & -1 & 0 & 0 \\
		0 & 1 & 0 & -1 & 0 \\
		-1 & -1 & 0 & 0 & 1
 \end{pmatrix*}
	\begin{pmatrix}
		k_1 x_1 x_3 \\
		k_2 x_2 x_3 \\
		k_3 x_1 \\
		k_4 x_2 \\
		k_5
	\end{pmatrix}
\end{align*}
As you can see this is the same as \ref{1st_example_dyn_system}, I just didn't use the notation " $\Gamma$ " then. This results in:
\[
	\begin{cases*}
		\dot{x}_1 = k_1 x_1 x_3 - k_3x_1  \\
		\dot{x}_2 = k_2 x_2 x_3 - k_4 x_2  \\
		\dot{x}_3 = -k_1 x_1 x_3 - k_2 x_2 x_3 + k_5 \\
	\end{cases*}
\]

\hfill\break
//TODO: poti sa scrii in general de ce ala un interaction graph din cartea aia de-ai gasit-o, si dupa sa zici ca poza aia de-o punem la phorolylation de-hpoysohp e un interaction graph pt networku asta: gen
\hfill\break