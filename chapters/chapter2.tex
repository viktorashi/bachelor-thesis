\chapter{Introduction to Chemical Reaction Networks}\label{chap:ch1}

In this first chapter we'll start with the concepts that apply to the broader range of \text{natural sciences}. We'll introduce and mathematically define

The following picture, that's also been shown by my Professor in his presentation of this topic might be a good introduction to this topic:

\hfill\break
//TODO: sa gasesti poza aia ca s-a cam luat de pe net
si sa mai scrii niste chestii din astea in mare de despre ce inseamna disciplina asta de prin paperuri sau istorie de prin ce carti gasesti sa faci o introdcuere pe bune, eu ma pun sa scriu speicfic de matele pe care il gasim
\hfill\break

A Chemical Reaction Network (CRN) consits of:

the set $S = \left\{ X_1, \dots, X_n \right\}, \left| S \right| = n$ of chemical species.

and $r$ reactions of the form:
\[
	\alpha_{1j} X_1 + \dots + \alpha_{n j} X_n \xrightarrow{k_j} \beta_{1j} X_1 + \dots + \beta_{n j}, \forall j = \overline{1,r}.
\]
Where $\alpha_{ij}$ is the stoichiometric coefficient of species $X_i$ in the reaction $j$.
The same goes for $\beta_{ij}$.
We denote the species' concentrations with $[X_1], \dots, [X_n]$.