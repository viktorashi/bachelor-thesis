\chapter{Existence and Absence of Hopf Bifurcation in Phosphorylation–Dephosphorylation CRN}

\section{What even \textbf{is} a bifurcation?}
To better generalise them, it'd be best to also define the notion of:

\begin{definition}
  \textbf{Invariant sets}.

  Taking the Auton. sys.
  \begin{equation}\label{invar_set_auton_sys}
    \dot{y} = f(y) , y(0) = y_0, y \in \mathbb{R}^n
  \end{equation}

  A stateset $S \subseteq \mathbb{R}^n$ of \ref{invar_set_auton_sys} is \textbf{invariant} if $\forall y_0 \in S, \forall t \geq 0: y(t) \in S$
\end{definition}

Now consider we have an system as above, except it now depends on an extra parameter $\mu$, which remains constant \textbf{during} our "experiment".
\begin{equation}\label{auton_parameter_sys_compact}
  \dot{y} = f_\mu(y).
\end{equation}

This could be, for example, the length of the pendulum $L$, in \ref{damped_pendulum}, or its damping coefficient $c$. Honestly generalising it even more, it could even be the gravitational constant $g$. Hence, this parameter can, of course be a \textbf{vector} of such parameters.

The way the parameter $\mu$ varies also induces variations in the topology and dynamics of the system. That's obvious since this is kind of the point of parameters in systems.

What can vary though can be trivial or more intereseting.

The positions of \textbf{equlibria}, for example can vary continuously during changes in $\mu$; that's usually normal. But what's more interesting is when these equilibria change their \textbf{stabilty} altogather, or even dissapear completely / appear out of thin-air.

That's called a \textbf{bifurcation}, and $\mu$, in this case is a \textbf{bifurcation parameter}.
\newpage
To better define and illustrate this for the 1-D case and give example of bifurcations for such dimension, we could instead write our vector field $f$ as depending on $\mu$ as well as the unknown function $y$.
\begin{equation}\label{eq:1-d_bif_sys}
  \dot{y} = f(y, \mu).
\end{equation}
And assume $f \in C^k(\mathbb{R} \bigtimes \mathbb{R}), k \geq 2 $ around $(0,0)$, and
\begin{equation}\label{bifurcation_priming}
  f(0,0) = 0, \quad \frac{\partial f}{\partial y}(0,0) = 0.
\end{equation}
By these assumptions we are basically "priming" the system for bifurcations, and here come the specific definitions for each particular type of bifurcation in 1-D.

\begin{definition}
  A \textbf{Saddle-node bifurcation}:
  for $f$ in \ref{bifurcation_priming}, add as well:
  \begin{equation*}
    \frac{\partial f}{\partial \mu}(0,0) =: a \neq 0, \quad \frac{\partial^2 f}{\partial y^2}(0,0) =:b \neq 0.
  \end{equation*}
\end{definition}
Then, for \ref{eq:1-d_bif_sys}, a saddle-node bifurcation accurs at $\mu = 0$, characterised by the following equivalances:

\rom{1}. For $ab < 0$ (resp. $ab> 0$), $\nexists y : f(y,\mu) = 0$ for $\mu < 0$ ( resp. $\mu > 0$)

\rom{2}. For $ab < 0$ (resp. $ab > 0$), $\exists y_+(\epsilon) \neq y_-(\epsilon) \implies y_\pm(\epsilon) : f(y_\pm(\epsilon), \mu) = 0, \epsilon = \sqrt{\abs{\mu}}$, for $ \mu > 0 $ (resp. $\mu < 0$ ), having opposing stabilities.

\begin{definition}
  \textbf{Pitchfork bifurcation}.

  For $f$ from \ref{bifurcation_priming}, assume as well that:
  $f \in C^k, k \geq 3$,
  \[
    f(-y, \mu) = -f(y,\mu)
  \]
  and,
  \[
    \frac{\partial^2 f}{\partial \mu \partial y}(0,0) =: a \neq 0, \quad \frac{\partial^3 f }{\partial y^3}(0,0)=: b \neq 0
  \]

  Now, a \textbf{pitchfork bifurcation} accurs for \ref{eq:1-d_bif_sys} at $\mu  =0$, char. by:

  \rom{1}. for $ab < 0$ (resp. $ab > 0$) $\exists! y = 0 :f(y,\mu) = 0 $ for $\mu < 0$ (resp. $\mu > 0$). $b < 0 \implies$ stable, $b > 0 \implies$ unstable.

  \rom{2}. for $ab < 0$ (resp. $ab > 0$) $\exists y = 0 : f(y , \mu) = 0$, as well as: $y_+(\epsilon) \neq y_-(\epsilon) \implies y_\pm(\epsilon) : f(y_\pm(\epsilon), \mu) = 0, \epsilon = \sqrt{\abs{\mu}}$  for $\mu > 0$ (resp. $\mu < 0$), for which $y_+(\epsilon) = -y_-(\epsilon)$, Both $y_-(\epsilon)$ and $y_+(\epsilon)$ have matching stabilities, whereas $y = 0$ has the opposite stability of them, given by $b < 0 \implies$ stable, $b > 0 \implies$ unstable.
\end{definition}
\newpage
\begin{definition}
  \textbf{Transcritical bifurcation}.

  For $f$ in \ref{bifurcation_priming}:
  \[
    \frac{\partial^2 f}{\partial \mu \partial y}(0, 0) =:a \neq 0, \quad \frac{\partial^2 f}{\partial y^2}(0,0) =: b \neq 0
  \]
  Then a \textbf{transcritical bifurcation} accurs for \ref{eq:1-d_bif_sys} at $\mu = 0$, char. by:

  \rom{1}.  $\exists y = 0 : f(y, \mu) = 0$  and  $ \exists y_0(\mu) : f(y_0(\mu), \mu) = 0 $ where $\mu \rightarrow u_0(\mu) \in C^m, m = k - 2$

  \rom{2}.  for $a\mu  < 0$ (resp. $a\mu > 0$) the trivial fixed point $y = 0 \implies$ stable (resp. unstable)
  and $u_0(\mu)$ has opposing stability.
\end{definition}

Finding bifurcations can also be done by looking at eigenvalues.

\section{What is a simple Hopf Bifurcation?}

A \textit{simple Hopf bifurcation} is a bifurcation in which a single complex-conjugate pair of eigenvalues of the Jacobian matrix crosses the imaginary axis, while all other eigenvalues remain with negative real parts. Such a bifurcation generates nearby oscillations or periodic orbits.

\subsection{Proving the existence of simple Hopf bifurcations}

\subsection{Ruling out simple Hopf bifurcations}

\subsection{Convex parameters}

\section{What is a Phosphorylation–Dephosphorylation CRN?}

\subsection{Cyclic and mixed distributive and processive Phosphorylation–Dephosphorylation CRN}