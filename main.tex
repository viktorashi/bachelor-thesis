%\documentclass[12pt]{scrreprt}
\documentclass[10pt]{report} 

% language may be romanian or english (default is english)
% type may be bachelor or master (default is bachelor)
\usepackage[language=english, type=bachelor]{style}

% sa desenez cerculetele alea
\usepackage{tikz}

% pt nr reale
\usepackage{amssymb}

% pt produsu cartezian
\usepackage{mathabx}



%\geometry{a4paper,top=2.5cm,left=3cm,right=2.5cm,bottom=2.5cm}
%in style
%controlling the appearance of your headers and footers
\usepackage{fancyhdr}
\pagestyle{fancy}
\lhead{}
\chead{}
\renewcommand{\headrulewidth}{0.1pt}
\renewcommand{\footrulewidth}{0.1pt}

%asta e destul de picky
\usepackage{witharrows}

% sa arate mai bine coadele
\usepackage{xcolor}
\usepackage{listings}
%New colors defined below
\definecolor{codegreen}{rgb}{0,0.6,0}
\definecolor{codegray}{rgb}{0.5,0.5,0.5}
\definecolor{codepurple}{rgb}{0.58,0,0.82}
\definecolor{backcolour}{rgb}{0.95,0.95,0.92}

\usepackage{amsthm}

% pt sistemu de ecuatii
\usepackage{amsmath}

% sa facem teorma gotica lmao
\usepackage{yfonts}

%Code listing style named "mystyle"
\lstdefinestyle{mystyle}{
  backgroundcolor=\color{backcolour}, commentstyle=\color{codegreen},
  keywordstyle=\color{magenta},
  numberstyle=\tiny\color{codegray},
  stringstyle=\color{codepurple},
  basicstyle=\ttfamily\footnotesize,
  breakatwhitespace=false,         
  breaklines=true,                 
  captionpos=b,                    
  keepspaces=true,                 
  numbers=left,                    
  numbersep=5pt,                  
  showspaces=false,                
  showstringspaces=false,
  showtabs=false,                  
  tabsize=2
}

%"mystyle" code listing set
\lstset{style=mystyle}

\begin{document}

\theoremstyle{definition}
\newtheorem{theorem}{\textswab{Theorem}}
\newtheorem{corollary}[theorem]{Corollary}
\theoremstyle{definition}
\newtheorem{definition}{Definition}

\specialization{[Secție]}
\title{[Titlu lucrare]}
\author{Stan Ioan-Victor}
\supervisor{[Grad, Lorand Gabriel Parajdi]}

\maketitle


\newpage
\thispagestyle{empty}
\mbox{}
\newpage
\pagenumbering{roman}

\cleardoublepage
ABSTRACT
\vspace{0.5cm}
\hrule
\vspace{0.5cm}
%\cleardoublepage

Abstract:
un abstract in engleza
one in romanian
un rezumat în limba engleză




cu prezentarea, pe scurt, a conținutului pe capitole, punând accent pe contribuțiile proprii și originalitate



\tableofcontents


\newpage
\pagenumbering{arabic}

\chapter{Introduction}

%\chapter*{Introducere}
\label{intro}

Life, is run by chemical reactions, which regulate, and propagate throughout the organism. These processes, from the beating of our hearts to the firing of our neurons, are modeled by the principles of dynamical systems. An important question is their oscillatory behavior. Can a network of chemical reactions act like a biological clock, producing sustained, periodic patterns? Or does it behave like a switch, capable of settling into multiple stable states? Here we'll discuss the former behaviour.

This thesis delves into the heart of this question by investigating the existence and absence of a specific mathematical phenomenon known as a Hopf bifurcation within a crucial class of biochemical systems: cyclic phosphorylation-dephosphorylation reaction networks. A Hopf bifurcation is a critical point at which a system's stable equilibrium gives way to sustained oscillations. Proving whether such a bifurcation can occur is paramount to determining if a given network has the intrinsic ability to generate periodic behavior.

To achieve this goal, this work embarks on a journey from fundamental principles to advanced application. We begin by establishing the mathematical language required to describe these systems, introducing the core concepts of Chemical Reaction Network (CRN) theory and the fundamentals of dynamical systems and ordinary differential equations. We will build the theoretical apparatus step-by-step, providing the reader with the geometrical intuition and analytical tools necessary to understand the conditions under which bifurcations arise.

Through researching this foundational knowleedge, I've stumbled around a fact I wasn't able to find \textbf{ proofs } on, so I tried making a short one myself: Demonstrating that the orbits of a dynamical system form a partition of its phase space using only elementary set theory instead of equivalence relations which I've seen before and frankly can't fully grasp.

We apply these tools to phosphorylation-dephosphorylation cycles and will develop and rigorous criteria, leveraging the properties of the Hurwitz matrix and convex parameters, to either rule out or confirm the possibility of Hopf bifurcations within these networks. Through detailed analysis, we will pinpoint initial conditions and parameter ranges that give rise to oscillatory dynamics.

Finally, "Open-CoNtRol" is presented. A web-based application developed to overcome the significant visualization and computational hurdles inherent in this field of study. This tool, which builds upon and contributes to established open-source scientific libraries, empowers researchers and students to numerically analyze, simulate, and visualize the complex behaviors of chemical reaction networks.

In essence,  by combining rigorous mathematical proof with the development of a practical software tool, it provides both the theoretical framework and the applied methodology for proving the existence and absence of Hopf bifurcations.

I've also had the opportunity of presenting this work at the Student Scientific Communication Session in Mathematics, on May 24th, 2025.
%\addcontentsline{toc}{chapter}{Introducere}
%\addcontentsline{toc}{chapter}{Introduction}

\chapter{Introduction to Chemical Reaction Networks}\label{chap:ch1}

In this first chapter we'll start with the concepts that apply to the broader range of \text{natural sciences}. We'll introduce and mathematically define

The following picture, that's also been shown by my Professor in his presentation of this topic might be a good introduction to this topic:

\hfill\break
//TODO: sa gasesti poza aia ca s-a cam luat de pe net
si sa mai scrii niste chestii din astea in mare de despre ce inseamna disciplina asta de prin paperuri sau istorie de prin ce carti gasesti sa faci o introdcuere pe bune, eu ma pun sa scriu speicfic de matele pe care il gasim
\hfill\break

A Chemical Reaction Network (CRN) consists of:

the set $S = \left\{ X_1, \dots, X_n \right\}, \left| S \right| = n$ of chemical species,

and $r$ reactions of the form:
\begin{equation}\label{mass-action_network}
	\alpha_{1j} X_1 + \dots + \alpha_{n j} X_n \xrightarrow{k_j} \beta_{1j} X_1 + \dots + \beta_{n j}, \forall j = \overline{1,r}.	
\end{equation}

Where $\alpha_{ij}$ is the stoichiometric coefficient of the reactant $X_i$ in reaction $j$.

The same goes for $\beta_{ij}$, but for the \textbf{produced} $X_i$ in reaction $j$.

$k_j$ are the reactions' rate constants.

We denote the species' concentrations with $[X_1], \dots, [X_n]$.

Now we may define the mass-action \textbf{reaction rate} of this particular reaction, which depends on the rate constant, the concentrations of reacting species and their molecularities.
\begin{equation}\label{reaction_rate}
	v = k_j [X_1]^{\alpha_{1j}} \dots [X_n]^{\alpha_{nj}}
\end{equation}

\hfill\break
//TODO poti sa mai zici si de alea de non-mass action si de unde a aparut si de unde a disparut sau cum se foloseste fiecare, si ala de mass action
\hfill\break
A pretty common example would be:
\[
	2H_2 + O_2 \xrightarrow{k} 2H_2O
\]
Where its reaction vector is:
\[
	\bordermatrix{\cr \cr
		H_2 & -2 \cr
		O_2 & -1 \cr
		H_2O & 2 \cr
	}
\]
with reaction rate $v(x) = k[H_2]^2[O_2]$. Or, by using the vector $x=(x_1, x_2, x_3, x_4)$: $v(x) = k x_1^2 x_2$.
So the reaction's dynamical system (defined more in detail in \ref{3.2.1}) is:
\begin{align}\label{1st_example_dyn_system}
	\frac{d}{dt}
	\begin{pmatrix*}
		x_1 \\
		x_2 \\
		x_3
	\end{pmatrix*} =
	v(x)
	\begin{pmatrix}
		-2 \\
		-1 \\
		2
	\end{pmatrix}
\end{align}
But if we have multiple such reactions we need to use their $\ldots$
\begin{definition}
	\textbf{Stoichiometric matrices.}
	\begin{align*}
		\Gamma_L, \Gamma_R \in \mathcal{M}_{n \bigtimes r}(\mathbb{N}) \\
		(\Gamma_L)_{ij} = \alpha_{ij} \text{ from \ref{mass-action_network}} \\
		(\Gamma_R)_{ij} = \beta{ij} \text{ from \ref{mass-action_network}}
	\end{align*}
	$\Gamma_L$ the \textbf{left} stoichiometric matrix, \\
	$\Gamma_R$ the \textbf{right} stoichiometric matrix, and \\
	$\Gamma = \Gamma_R - \Gamma_L$ : the \textbf{full} stoichiometric matrix.
\end{definition}
So constructing their equivalent of \ref{1st_example_dyn_system}:
\begin{align}
	\frac{d}{dt}
	\begin{pmatrix*}
		x_1 \\
		\vdots \\
		x_n
	\end{pmatrix*} = \Gamma
	\begin{pmatrix*}
		v_1(x)	 \\
		\vdots \\
		v_r(x)
	\end{pmatrix*}
	\text{ or, even more shorthand:  }
	\frac{dx}{dt} = \Gamma v(x).
\end{align}
Where $v(x) \in \mathcal{M}_{r \bigtimes 1}$ is the column vector of all reaction rates for the reactions.

\hfill\break
//TODO: poti sa scrii in general de ce ala un interaction graph din cartea aia de-ai gasit-o, si dupa sa zici ca poza aia de-o punem la phorolylation de-hpoysohp e un interaction graph pt networku asta: ge
\hfill\break
\chapter{Dynamical Systems}
\label{chap:ch3}

\section{Defining Ordinary Differential Equations}
\begin{definition}
	\textit{An \textbf{ordinary differential equation} is an equation of an unknown function of \textbf{one} variable. This can be expressed as a function \textbf{of} this unknown function, its corresponding variable and its various derivatives}.
\end{definition}
Its general form looks something like:
\begin{equation}\label{ODE}
	F(t,y(t),y'(t),....,y^{(n)}(t))=0,
\end{equation}
Where $y(t)$ is the unknown function of independent variable $t$ and $F : \Omega \rightarrow \mathbb{R},  \\ \Omega \subseteq \mathbb{R}^{n+1}$.

This would be what's called the implicit form of the differential equation.

\begin{definition}
	\textit{The \textbf{order} of an ODE is the highest order of the derivative present in the equation.}
\end{definition}
In our case, the order is $n$.
If, however $F$ satisfies the regularity condition \textbf{of the implicit function theorem} then the equation can be written in a much more digestible form.

The \textbf{implicit function theorem} allows one to convert a relation (implicit form) to functions of some variables. These functions may not be unique, but together, their graph may locally satisfy the relation.

As an example: The relation for a cardioid cannot be expressed as a sole function, we'd instead need the union of the graph of two separate functions for expressing it.
(si aici gen faci tu niste graphuri cu maple sau alte d-astea vezi cum faci virtual machineu ala sau mergi la faculta la un pc cu maple si iti faci lista cu ce vrei sa graphuiesti)

\textbf{Reminder}: A function $f(x)$ is continuously differentiable if $\exists f'(x)$ and $f'(x) \in C^n$ where $n>=0$

\begin{theorem}
	Let $F:D \subseteq \mathbb{R}^{n+1}\rightarrow\mathbb{R}$ be a continuously differentiable function with the relation that $F(t,y(t),y'(t),....,y^{(n)}(t))=0$. Let us express a point in the set $\mathbb{R}^{n+1} =\mathbb{R}\bigtimes\mathbb{R}^n$ as (t, \textbf{Y}) = $(t, y,y', \dots, y^{ (n) })$ and fix one such point s.t. $F(t, \textbf{Y})=0$.
	So $\exists \square_{(t, \textbf{Y})} \ni U \bigtimes V \subseteq D$ s.t. $F \in C^1(U\bigtimes V)$ and $\frac{\partial F}{\partial y}(t, \textbf{Y}) \neq 0.$ Then this means $ \exists \square_{t} \ni U_0 \subseteq U$,
	$\square_{\textbf{Y}} \ni V_0 \subseteq V$ and a function $f : U_0 \rightarrow V_0$ s.t. $f(t) = \textbf{Y}$ and $F(t,f(t))=0, \forall t \in U_0; f \in C^1(U_0)$ and $\frac{\partial f}{\partial t_i} = - \frac{\frac{\partial F}{\partial t_i}(t,f(t))}{\frac{\partial F}{\partial y}(t,f(t))}, \forall i = \overline{1,n} , \forall t \in U_0$ and $F \in C^1(U \bigtimes V),K \in N^* \Rightarrow f \in C^K(U_0).$
\end{theorem}

\hfill\break
//TODO : proof: trust me bro
\hfill\break

So restricting the domain $\Omega$ to one that allows the implicit form to be represented as a function of the form
\begin{equation}\label{implicit_func_theorem}
	y^{(n)}(t)=f(t,y(t),y'(t),...,y^{(n-1)}(t))
\end{equation}
would yield what's called the \textbf{explicit} form of the ODE.

A \textbf{solution} is an expression of the unknown function $y(t)$ which satisfies the relation
$y^{(n)}(t) = f(t,y(t),y'(t),\dots,y^{(n-1)}(t))$. A \textbf{general solution} includes all of these functions and usually has some constants of integration in the expression, while a \textbf{particular solution} doesn't have them.

A particular solution is usually found if we also have some initial conditions given for the unknown function, like $y(a)=b, y'(a)=c$ for some $a,b,c \in \mathbb{R}$.
Or, more formally:

\begin{definition}
	A function $y:I \rightarrow \mathbb{R}$ is a solution of the equation \ref{ODE} if the following conditions are met:

	1. $I \subseteq \mathbb{R}$ is nondegeneate interval. ($|I|>1$)

	2. $y \in C^n(I)$ and $(t,y(t), y'(t), \dots, y^{(n)}(t)) \in D_f, \forall t \in I$.

	3. $y^{(n)}(t)= f(t,y(t),y'(t),\dots,y^{ (n-1) }(t)), \forall t \in I$. ( \ref{implicit_func_theorem} is satisfied)

\end{definition}

\section{Forming Ordinary Differential Equation systems}
We may take a sequence of such first order ordinary differential equations to form a system:
\begin{equation}\label{3.2.1}
	\begin{cases}
		y'_1(t) = f_1(t,y_1(t),y_2(t),\dots,y_n(t)) \\
		y'_2(t) = f_2(t,y_1(t),y_2(t),\dots,y_n(t)) \\
		\vdots                                      \\
		y'_n(t) = f_n(t,y_1(t),y_2(t),\dots,y_n(t))
	\end{cases}
\end{equation}

Constructing $y: D \subseteq \mathbb{R} \rightarrow \mathbb{R}^n, f : D_f \subseteq \mathbb{R}^{n+1} \rightarrow \mathbb{R}^n$ and denoting \\
$y(t)= (y_1(t), \dots, y_n(t))$ and $f(t,y_1(t),\dots,y_n(t)) = (f_1,\dots,f_n)$ we get what is called the \textbf{vector form} of the system \ref{3.2.1}.

\begin{equation}\label{3.2.2}
	y'(t) = f (t, y(t))
\end{equation}

\begin{definition}
	For a function $y:D \subseteq \mathbb{R} \rightarrow \mathbb{R}^n$ , $|D| > 1$, if $y \in C^1(D,\mathbb{R}^n)$, and $y'(t) = f(t,y(t)), \forall t \in D \Rightarrow$ $y$ is a \textbf{solution of the system} \ref{3.2.2}.
\end{definition}

\begin{definition}
	An \textbf{autonomous equation} is a differential equation that does not explicitly depend on time, but is instead only defined by the relation between the function itself and its derivatives, so an equation of the form
	\begin{equation}\label{eq:3.2.3}
		f(y^{(n)}(t), y^{(n-1)}(t), \dots, y(t))= 0
	\end{equation}
	That is not to say, the function itself does not depend on time, but rather the independent variable $t$ does not explicitly appear in the equation.
\end{definition}

One more famous such example is the equation for the damped oscillator
\begin{equation}\label{eq:3.2.4}
	\ddot{x} + 2\gamma\dot{x} + \omega^2x = 0.
\end{equation}

Where:
\begin{itemize}
	\item $\gamma$ is the damping factor, meaning how quickly the oscillations of the system decay due to loss of momentum (caused by air resistance or friction)

	\item $\omega$ represents the angular (natural) frequency the oscillator would have, were it not to be affected by damping, defined by
		$\omega = \sqrt{\frac{k}{m}}$; where $k$ is the \textbf{elastic stiffness coefficient} of the spring and $m$ the mass.
\end{itemize}

In addition, a frequently used notation in physics is
$\dot{x} = \frac{d}{dt}x$ , $\ddot{x}=\frac{d^2}{dt^2}x$

That equation is a \textbf{linear} homogenous equation with constant coefficients (hence autonomous), which is pretty nice to work with. Their general form would be:
\begin{equation}\label{lin_hom_eq_const}
	a_0 x + a_1\dot{x}+ \dots +a_{n} x^{(n)} = 0
\end{equation}

Finding exact solutions, though, can be done by using some nice properties of Euler's exponential function $e^{rt}$.

Introducing the notion of \textbf{operators} would help in writing this next section.

\begin{definition}
	$D = \frac{d}{dt}$ is the \textbf{derivation operator}:
	$D^n = \frac{d^n}{dt^n}$, so $D^nu = D^n(u) =\frac{d^nu}{dt^n}$ where $u=u(t)$ and (related to \ref{lin_hom_eq_const}): \par
	\hspace{20pt} $L = a_0 + a_1D + \dots + a_nD^n$ \par
	\hspace{20pt} is the \textbf{linear differential operator of order n}. Meaning \ref{lin_hom_eq_const} $\iff Lx=0$.
\end{definition}

We now can; just by pure Ansatz; assume the solutions take the form $e^{rt}$ for some (yet unknown) $r$'s. Because the exponential is so easy to work with when it comes to taking derivatives - a.k.a. : $D^n(e^{rt})=r^ne^{rt}$, so its derivatives are proportional to itself, meaning no new $t$'s are introduced - substituting back into \ref{lin_hom_eq_const} yields
\begin{gather*}
	Le^{rt} = a_0 e^{rt} + a_1 De^{rt} + \dots +a_{n} D^ne^{rt} = 0      \\
	\Updownarrow \\
	a_0 e^{rt}+a_1re^{rt} + \dots + a_nr^ne^{rt} =0 \\
	\Updownarrow \\
	a_0 + a_1r + \dots + a_nr^n =0
\end{gather*}
The left-hand-side is the \textbf{characteristic polynomial} of the linear equation.
\begin{equation}\label{lin_eq_char_poly}
	p(r) = a_1r + \dots + a_nr^n
\end{equation}
Notice that $L = p(D)$. \\
So now it all comes down to finding its roots $r$.

Case \rom{1}: A \textbf{unique, real} root found:

now $e^{rt}$ is a \textbf{fundamental solution} for $Le^{rt}=0$

Case \rom{2}: A \textbf{unique, complex} root found:

say it has the form $r = \lambda \pm i\mu, \mu \neq 0$. \\
By Euler's formula:
\begin{equation}\label{euler}
	e^{rt} = e^{(\lambda + i\mu)t} = e^{\lambda t} \cdot e^{i \mu t}  =  e^{\lambda t} \cdot [\text{cos}(\mu t) + i \text{sin}(\mu t)]
\end{equation}

So $e^{\lambda t}\text{cos}(\mu t)$ and $e^{\lambda t}\text{sin}(\mu t)$ are both fundamental solutions for $Lx=0$.

\textbf{Proof}: \par
Since $a_i \in \mathbb{R}, \forall i=\overline{1,n}$ in $p \implies p(\overline{r})=\overline{p(r)}, \forall r \in \mathbb{C}$, hence, just like 2AM coffee and the exam session, complex roots of $p$ always come in pairs. They both satisfy:
\begin{equation}\label{lin_op_solution}
	Le^{(\lambda \pm i \mu)} = 0
\end{equation}

But since our $x : \mathbb{R} \rightarrow \mathbb{R}$ we're looking for \textbf{real} valued solutions.
Because $\mu \in \mathbb{R}$ the following properties hold equivalently for $\lambda + i\mu$ and for $\overline{\lambda + i\mu}=\lambda - i\mu$ so I'll only write $\lambda \textbf{+} i \mu$ for simplicity. By \ref{euler}:
\begin{equation}\label{eulers_real_and_imaginary}
	\text{Re}(e^{rt})=e^{\lambda t} \cdot \text{cos}(\mu t), \hspace{2cm}    \text{Im}(e^{rt}) = e^{\lambda t} \cdot \text{sin}(\mu t).
\end{equation}
But (\textbf{reminder})
\begin{gather}\label{real_imag_another_way_to_write}
	\text{Re}(c)= \frac{c+\overline{c}}{2}, \hspace{2cm}
	\text{Im}(c) = \frac{c-\overline{c}}{2i}
\end{gather}
\begin{gather*}
	\forall c \in \mathbb{C}
\end{gather*}

\newcommand\firstConclusion{\stackrel{\mathclap{\normalfont\mbox{\ref{eulers_real_and_imaginary}, \ref{real_imag_another_way_to_write}}}}{=\joinrel=\joinrel=\joinrel=\joinrel=\joinrel=}}

\newcommand\byDistributivity{\stackrel{\mathclap{\normalfont\mbox{dist.}}}{=\joinrel=\joinrel=}}

\newcommand\operatorSatisfy{\stackrel{\mathclap{\normalfont\mbox{\ref{lin_op_solution}}}}{=\joinrel=}}

Using these three conclusions we get:
\begin{align*}
	L(e^{\lambda t} \cdot \text{cos}(\mu t)) \firstConclusion
	L\left(\frac{e^{rt}+e^{\overline{r}t}}{2} \right) \byDistributivity
	\frac{1}{2} \left( Le^{rt} + Le^{\overline{r}t} \right) \operatorSatisfy 0 , \\
	L(e^{\lambda t} \cdot \text{sin}(\mu t)) \firstConclusion
	L\left(\frac{e^{rt}-e^{\overline{r}t}}{2i} \right) \byDistributivity
	\frac{1}{2i} \left( Le^{rt} - Le^{\overline{r}t} \right) \operatorSatisfy 0 ,
\end{align*}

In conclusion, if $Le^{ct}=0$ for some $c \in \mathbb{C} \implies L\text{Re}(e^{ ct })= L\text{Im}(e^{ct})=0, \\
\text{for } \text{Re}(e^{ct}), \text{Im}(e^{ct}) \in \mathbb{R}$.
Moreover, the same can be said for $e^{\overline{c}t}$. \qed

Now something let's say... \textbf{fundamental} (hehe) about these solutions, is they have to be linearly independent to each other. Meaning the only solution for
\begin{equation}\label{liniar_independece}
	c_1 x_1+c_2 x_2 + \dots + c^n x^n = 0
\end{equation}
is the trivial solution $c_1=c_2=\dots=c_n=0$ where $x_i$ are the fundamental solutions of the system.

So the reason I've emphasized "\textbf{unique}" in the first 2 cases was because if we're to apply those methods for \textbf{repeating} roots, meaning the ones with multiplicity $k>1$, the resulting solutions wouldn't hold \ref{liniar_independece} because they would in fact be the same!

To counteract this issue, the clever people over at the 1748 Berlin Academy (just Euler) came up with yet another (easily provable) Ansatz: just multiplying by powers of $x$.

A root with multiplicity $k$ would introduce yet another $k$ -for real- or $2k$ - for complex - fundamental solutions to this equation, so to finish what we've started:

Case \rom{3}: \textbf{Repeated, real} roots $r$ with multiplicity $k>1$ yield the fundamental solutions
\[
	e^{rt}, xe^{rt}, \dots,x^{k-1}e^{rt}.
\]

These can be verified that they do, in fact satisfy \ref{liniar_independece}. So can:

Case \rom{4}: \textbf{Repeated, complex} roots $r$ with multiplicity $k>1$. These introduce the pairs:
\begin{align*}
	e^{at}\text{cos}(bt), \ \  e^{at}\text{sin}(bt)  \\
	xe^{at}\text{cos}(bt), \ \ xe^{at}\text{sin}(bt) \\
	\vdots                                           \\
	x^{k-1}e^{at}\text{cos}(bt), \ \ x^{k-1}e^{at}\text{sin}(bt)
\end{align*}

So $2k$ new solutions.

\textbf{The superposition principle}

Bernoulli (to the disbelief of Euler and Langrange) told us that the complex behaviour of vibrating strings can be modeled by separating their motion into well-defined and easy to compute simple waves with known frequencies and amplitudes, which can later be \textbf{superposed} with one-another to form the complete system.

This principle can be applied to \textit{our} easier to compute \textbf{fundamental solutions} ($x_1,\dots,x_n$) in relation to the system formed by our linear operator $Lx =0$.

Since it's a linear operator, additivity and homogeneity apply:
\begin{gather*}
	L(x_1+x_2) = Lx_1 + Lx_2\\
	aLx = L(ax) , \forall a \in \mathbb{C}
\end{gather*}
where $x,x_1,x_2$ are fundamental solutions of the system.

Hence the complete, final, \textbf{general} solution for \ref{lin_hom_eq_const} are linear combinations of all our previously found fundamental solutions:
\[
	x(t) = c_1 x_1 + \dots + c_nx_n
\]
for some constants $c_i$.
\newpage
\section{Formally defining and characterizing dynamical systems}
The reason why autonomous equations are so useful to us is because of the ease with which we can visualize the \textbf{evolution} of a system of such equations.
Take, for example:
\begin{definition}
	A \textbf{system of first-order autonomous equations} is a system of equations of the form
	\begin{equation}\label{fo_system_auton_eq}
		\begin{cases}
			\dot{y}_1 = f_1(y_1,\dots, y_n) \\
			\dot{y}_2 = f_2(y_1,\dots, y_n) \\
			\vdots                          \\
			\dot{y}_n = f_n(y_1,\dots, y_n) \\
		\end{cases}
	\end{equation}
\end{definition}

\begin{definition}
	We can construct the \textbf{phase space} of the \ref{fo_system_auton_eq} dynamical system by:

	\begin{enumerate}
		\item Representing all possible values of $y_i$, $ i = \overline{1,n}$ as points in Euclidean space $S \subseteq \mathbb{R}^n$

		\item Constructing a function $f : S \rightarrow S, f(y_1,y_2,\dots,y_n) = (\dot{y}_1,\dot{y}_2,\dots,\dot{y}_n)$ that attaches to each point in $S$ a vector representing the direction in which to go such that the relation \ref{fo_system_auton_eq} is satisfied.
	\end{enumerate}
\end{definition}

Thus, the \textbf{phase space} of the system is a vector field where $f$ is a continuous vector valued function.

So keeping this in mind, we may form a formal definition:

\begin{definition}
	A \textbf{dynamical system} is a tuple $(T,S,\Phi)$ where:

	\begin{itemize}
		\item $T$: part of the $(T,+)$ monoid of all possible $time$ moments of our system
		\item $S$: is the set of all possible states, and
		\item the function: $\Phi : U \subseteq (T \times S ) \rightarrow S$ where values of the second component span all of $S$, meaning $proj_2(U) = S$.
	\end{itemize}
	In order to give the necessary properties of $\Phi$ we will need to also define:
	\begin{definition}\label{dyn_sys_orbit_flow_etc_def}
		\textbf{A lot of notions about dynamical systems.}

		$I(s):= \{ t \in T | (t,s) \in U \}, \forall s \in S$

		so $\Phi$ has the properties that:

		$\Phi(0,s) = s$

		$\Phi(t_2,\Phi(t_1,s)) = \Phi(t_1+t_2,s); \forall t_1, t_1+t_2 \in I(s), t_2 \in I(\Phi(t_1,s)), s \in S$.

		We call such a function $\Phi$ the \textbf{evolution function} of the dynamical system, mapping every state and $time$ after which that specific state is found, to another unique state.
	\end{definition}

	If we take a specific state $s$ as an initial constant, we can form the function

	$\Phi_s : I(s) \rightarrow S$
	called the \textbf{flow} through $s$, whose image:

	$\gamma_s:=\{\Phi_s(t) | t\in I(s) \}$
	is the \textbf{orbit} through $s$,
	which is a curve of parameter $t \in I(s)$ representing the values the dynamical system will take throughout the experiment's timespan, given the initial condition (state) $s$.
\end{definition}

Since a point only belongs to one orbit, if we form the set
$\Gamma_{U,\Phi}:=\{ \gamma_s | s \in S \}$
of all orbits of the dynamical system $(U,\Phi)$, with $U\subseteq (T \times S)$ and $\Phi$ the evolution function, it follows that:
$\forall \gamma_1,\gamma_2 \in \Gamma_{U, \Phi}, \gamma_1 \bigcap \gamma_2 = \emptyset$, meaning the set of orbits forms a partition of the phase space $S$

\textbf{Proof:}
Conditions for partitions:
\par \textbf{Reminder:} A partition $\Gamma_U$ of a set $U$ is a set of non-empty disjoint subsets of $U$:
\[
	\bigcup_{\gamma\in \Gamma_U} \gamma = U
\]

So we need to prove:

\hfill\break
//TODO: te apuci de asta mai incolo \par
\hfill\break

\rom{1}. $\gamma \subseteq U, \forall \gamma \in \Gamma_{U,\Phi}$ \par
\rom{2}. $ \gamma \neq \emptyset, \forall \gamma \in \Gamma_{U,\Phi}$ \par
\rom{3}. $\gamma_1 \bigcap \gamma_2 = \emptyset, \forall \gamma_1, \gamma_2 \in \Gamma_{U,\Phi}$ \par
\rom{4}. $\bigcup_{\gamma\in \Gamma_U} \gamma = U$ \par

\hfill\break
//TODO: oki, deci aici zici cum se rezolva o ecuatie autonoma liniara in mod normal si despre solutiile fundamentale si independenta si dependa lor and stuff si dupa dai un mod echivalnt de a le rezolva cu metoda de mai jos care face un sistem din ele
\hfill\break

We may, in fact construct an equivalent system
\[
	\begin{cases}
		\dot{y}_1 = f_1(y_1,\dots, y_n) \\
		\dot{y}_2 = f_2(y_1,\dots, y_n) \\
		\vdots                          \\
		\dot{y}_n = f_n(y_1,\dots, y_n) \\
	\end{cases}
\]

Starting from any autonomous differential equation
\[
	f(y^{(n)}(t), y^{(n-1)}(t), \dots, y(t))= 0
\]
via a technique which associates to each respective derivative a state unknown (variable).
This system's form depends on the solvability with respect to the highest derivative $y^{(n)}$.
The first form may be obtained if the above equation \textbf{can} be solved in terms of $y^{(n)}$, meaning it can be re-written as
\begin{gather*}
	f(y^{(n)}(t), y^{(n-1)}(t), \dots, y(t))= 0 \\
	\Updownarrow \\
	y^{(n)}  = G(y^{(n-1)}(t), \dots, y(t))
\end{gather*}

We first take each state unknown to be equal to their respective derivatives, so:
\begin{gather*}
	y_1 = y     \\
	y_2 = \dot{y} \\
	\vdots \\
	y_n = y^{(n-1)}
\end{gather*}

Then we take the derivatives with respect to time of each of these equations, so starting with $\dot{y}_1 = \dot{y}$, but $\dot{y} = y_2$, so $\dot{y_1}=y_2$. Continuing on like that, in a sort of diagonal fashion for all but the last state variable, we get:
\begin{gather*}
	\dot{y}_1 = y_2 \\
	\dot{y}_2 = y_3 \\
	\vdots \\
	\dot{y}_{n-1} = y_n
\end{gather*}
and this is where that beginning "solvability" part comes into play. The very last one will be exactly its solution defined earlier:
\[
	\dot{y}_n = \frac{d}{dt}y^{(n-1)} = y^{(n)}  = G(y^{(n-1)}(t), \dots, y(t))
\]

So, putting this all together we can get the equivalent system of the autonomous differential equation
\[
	\begin{cases}
		\dot{y}_1 = y_2     \\
		\dot{y}_2 = y_3     \\
		\vdots              \\
		\dot{y}_{n-1} = y_n \\
		\dot{y}_n  = G(y^{(n-1)}(t), \dots, y(t))
	\end{cases}
\]

Now, if instead of an actual function $G$ we only get an implicit form for a solution (\ref{implicit_func_theorem} isn't satisfied), we could write the original equation in terms of these new state variables:
\begin{gather*}
	f(y(t), \dot{y}(t),\dots,y^{(n-1)},y^{(n)})= 0 \\
	\Updownarrow \\
	f(y_1,\dots,y_n,\dot{y}_n) = 0
\end{gather*}

So the system would be:
\[
	\begin{cases}
		\dot{y}_1 = y_2     \\
		\dot{y}_2 = y_3     \\
		\vdots              \\
		\dot{y}_{n-1} = y_n \\
		f(y_1,\dots,y_n,\dot{y}_n) = 0
	\end{cases}
\]

This is a differential-algebraic system of equations (DAE), which is beyond the scope of this thesis but figured I'd cover all cases (this one's usually solved via numerical methods).

Okay, but now since we know how to form systems from any autonomous equation \dots how does that make our lives easier in any way?

Let's consider \ref{lin_hom_eq_const}, the equation we worked with previously. Now we can express it differently:

We could first divide everything by $a_{n}$ so everything looks a bit better.
\begin{gather*}
	\left. a_0 x + a_1\dot{x}+ \dots +a_{n} x^{(n)} = 0 \middle| \frac{1}{a_{n}} \right. \\
	\Updownarrow \\
	b_0 x + b_1\dot{x} + \dots + b_{n-1} x^{(n-1)} + x^{(n)} = 0
\end{gather*}

And now apply the aforementioned technique:
\begin{gather*}
	x_1 = x     \\
	x_2 = \dot{x} \\
	\vdots \\
	x_n = x^{(n-1)}
\end{gather*}

then:
\[
	\begin{cases}
		\dot{x}_1 = x_2     \\
		\dot{x}_2 = x_3     \\
		\vdots              \\
		\dot{x}_{n-1} = x_n \\
		\dot{x}_n  = x^{(n)} = -b_0x - b_1\dot{x} - \dots - b_{n-1} x^{(n-1)} = -b_0 x_1 - b_1 x_2 - \dots - b_n x_{n-1}
	\end{cases}
\]

Waitt, I saw this somewhere in year 1, semester 1 during Linear Algebra (which I definitely didn't have to re-take twice)

That is the same as:
\begin{align*}
	\begin{bmatrix}
		\dot{x}_1     \\
		\dot{x}_2     \\
		\dot{x}_3     \\
		\vdots        \\
		\dot{x}_{n-1} \\
		\dot{x}_n
	\end{bmatrix} =
	\bordermatrix{ & x_1    & x_2    & x_3    & \ldots & x_n \cr
		& 0      & 1      & 0      & \ldots & 0 \cr
		& 0      & 0      & 1      & \ldots & 0 \cr
		& 0      & 0      & 0      & \ldots & 0 \cr
		& \vdots & \vdots & \vdots & \ddots & \vdots \cr
		& 0      & 0      & 0      & \ldots & 1 \cr
	& -b_0   & -b_1   & -b_2   & \ldots & b_n }
	\begin{bmatrix}
		x_1     \\
		x_2     \\
		x_3     \\
		\vdots  \\
		x_{n-1} \\
		x_n
	\end{bmatrix}
\end{align*}

Writing it in compact form:
\[
	\dot{x} = A x
\]
With $x, \dot{x}$ 2 $n$ sized column vectors and $A$ an $n \bigtimes n$ matrix.

Now notice that
\[
	\text{det}(A- r I_n)
\]

Is precisely the char. poly. \ref{lin_eq_char_poly}. So the $r$ solutions for
\[
	\text{det}(A - r I_n) =0
\]
(a.k.a the eigenvalues of the A matrix) are also the roots of \ref{lin_eq_char_poly}. So solving for them would obtain the solutions
\[
	x_i = e^{rt}x_i(0)
\]
for $i = \overline{1,n}$.

A much simpler example of a system that can be solved by not even having to make such computations is that of a \textbf{decoupled} dyamical system. Alas one where each state variable only depends on itself. Common examples that come to mind would be the concentration of a homogenous substance in an isolated tank while under the influence of heat transfer. This wouldn't really be a "system" so we could take multiple such tanks in a lab, where they are indeed sepparated from one-another but still form the system.
\begin{equation*}
	\begin{cases}
		\dot{x}_1 = \lambda_1 x_1 \\
		\dot{x}_2 = \lambda_2 x_2 \\
		\vdots                    \\
		\dot{x}_n = \lambda_n x_n
	\end{cases}
\end{equation*}

The system's matrix $D$ is now:
\begin{equation*}
	D =
	\begin{bmatrix}
		\lambda_1 & 0         & 0         & \dots  & 0         \\
		0         & \lambda_2 & 0         & \dots  & 0         \\
		0         & 0         & \lambda_3 & \dots  & 0         \\
		\vdots    & \vdots    & \vdots    & \ddots & \vdots    \\
		0         & 0         & 0         & \dots  & \lambda_n
	\end{bmatrix}
\end{equation*}

We notice that our solutions will now take the form
\[
	x = e^{Dt}x(0)
\]

where $\cdot^A : \mathbb{C} \rightarrow \mathbb{C}^{n \bigtimes m}$ for an $A = (a_{ij}) \in \mathbb{C}^{n \bigtimes m}$ is defined as:
\[
	\alpha^A = (\alpha^{a_{ij}}_{ij})
\]

so
\begin{align*}
	\underline{x}(t) =
	\begin{bmatrix}
		e^{\lambda_1} & 0             & 0             & \dots  & 0             \\
		0             & e^{\lambda_2} & 0             & \dots  & 0             \\
		0             & 0             & e^{\lambda_3} & \dots  & 0             \\
		\vdots        & \vdots        & \vdots        & \ddots & \vdots        \\
		0             & 0             & 0             & \dots  & e^{\lambda_n}
	\end{bmatrix}
	\underline{x}(0)
\end{align*}
Where I use $\underline{x}$ to denote the \textbf{vector form} of a the state-space vars.
In conclusion, it'd be ideal for us to work with such \textbf{diagonal} matrices.

There are, however situations when we don't have that luxury, so we could instead \textbf{transofrm} the way we write our $x$'s (and hence, our transformation as well) in order for us \textbf{diagonalize} our system. This would decouple the system and simplify the computations.

This is pretty straight-forward:

Starting from the system
\begin{equation}\label{diag_matrix_init_sys}
	\dot{x} = A x
\end{equation}
we need to find a transformation $T$ from our coordinate system
\begin{equation}\label{diag_trans_rel}
	x = Tz
\end{equation}
s.t. our dynamics are decoupled (the new transformation matrix $D$ is diagonal)
\[
	\dot{z} = Dz
\]
Derivating \ref{diag_trans_rel} and substituting into \ref{diag_matrix_init_sys}:
\begin{gather*}
	\left. x = Tz \middle| \frac{d}{dt} \right. \iff
	\dot{x} = T\dot{z} = Ax = ATz \Rightarrow \dot{z} = T^{-1} A T z = Dz \\
	\text{Which allows us to find an invertible } T \text{ s.t.} \\
	\text{the diagonal matrix } D \text{ is the conjugate of (similar to) } A \text{ w.r.t. } T \\
	\left. T^{-1} A T z = Dz \middle| \frac{T}{z}
	\right. \\
	\Updownarrow \\
	A T = T D.
\end{gather*}

Everything comes down to solving the above eigenvalue problem.

As stated in the above sections: $diag(D) = \{ \lambda | det(A - \lambda I_n) = 0 \}$ and it can also be found that the columns of $T$ are $A$'s corresponding eigenvectors $\xi_i$.

After computing $A$'s eigenvalues, we've essentially also found $T$, now
\[
	D = T^{-1} A T
\]
is used to find the diagonal matrix which will decouple our dynamics in the new coordinate system $z = T^{-1} x$; $\dot{z} = D z$.

Transforming back into our world, the solutions (in vector form) are:
\begin{gather*}
	\underline{z}(t) = e^{Dt}\underline{z}(0) \\
	\Updownarrow \\
	T^{-1} \underline{x}(t) = e^{Dt} T^{-1} \underline{x}(0) \\
	\Updownarrow \\
	\underline{x}(t) = T e^{Dt} T^{-1} \underline{x}(0). \\
	\Updownarrow \\
	\underline{x}(t) = T e^{Dt} \underline{z}(0). \\
	\Updownarrow \\
	\underline{x}(t) = T \underline{z}(t).
\end{gather*}
Now we know the equations have the form
\[
	x_i = T z_i(t), i = \overline{1,n}.
\]
Where $T$ is the $n \bigtimes n$ matrix whose columns are $A$ (not necesarily linearly independent) eigenvectors.
\begin{equation*}
	\begin{bmatrix*}
		\vdots & \vdots & \dots & \vdots \\
		\xi_1 & \xi_2 & \dots & \xi_n \\
		\vdots & \vdots & \dots & \vdots \\
	\end{bmatrix*}
\end{equation*}
Boom! You're golden.

Now the transformation only scales the basis vectors in our new coordinate system and we can use this to our advantage to find solutions in that system, then turn them back into our original system to see which values they correspond to.
\vskip\bigskipamount

Okay but what if my system is not linear? \textbf{Then} how will forming its corresponding system help me?

\vskip\bigskipamount
Si dai exemplul cu pendumulul, reintroducand idea de phase space si cum poti sa-l formezi tu acum folosind metoda de mai sus (o sa-i dau un nume mai incolo s-o referentiezi)
\vskip\bigskipamount

Taking, in particular, a more complex autonomous differential equation; the one for the damped pendulum:

\begin{equation}\label{damped_pendulum}
	\ddot{\theta} +\mu\dot{\theta} + \frac{g}{L}\sin(\theta) = 0
\end{equation}
where:  \par
$
\left.
\begin{array}{l}
	\theta : \text{angular position}       \\
	\dot{\theta} : \text{angular velocity} \\
	\ddot{\theta} : \text{angular acceleration}
\end{array}
\right\}
\text{all functions of } t \text{ (time)}
$

\ \ $\mu = \frac{c}{mL}$ : dampening factor \par
\ \ $c$ : the dampening coefficient due to the resistive forces \par
\ \ $m$: mass at the end of the pendulum \par
\ \ $g \approx 9.81 \frac{m}{s^2}$: earth's gravitational acceleration \par
\ \ $L$: the pendulum's length \par

We can express $\theta$ and $\dot{\theta}$ as two separate and independent variables in a 2D space of points \par $(\theta,\dot{\theta}) = (\theta,\omega)$.
And attach to each point the vector formed from the derivatives of each component, which would be (according to \ref{damped_pendulum}):

\begin{align}
\frac{d}{dt}
\begin{bmatrix}
	\theta \\
	\dot{\theta}
\end{bmatrix} =
\begin{bmatrix}
	\dot{\theta} \\
	\ddot{\theta}
\end{bmatrix} =
\begin{bmatrix}
	\omega \\
	-\mu \cdot \omega - \frac{g}{L}\sin(\theta)
\end{bmatrix}
\end{align}

This would in turn form the phase portrait of the dynamical system, described by a system of first-order autonomous differential equations:
\[
\begin{cases}
	\dot{\theta}  = \omega \\
	\dot{\omega} = -\mu \cdot \omega - \frac{g}{L}\sin(\theta)
\end{cases}
\]

\vskip\bigskipamount

//TODO: faci tu dupa aici un plot in matlab frumusel
si dupa mai gasesti un exemplu mistio in 3d

arat dupa ce faci toate demonstratiile pentru n dimensiuni, niste poze si cazuri particulare in 2,3 dimensiuni, cv ce se paote vedea cu poze

arata in state spaceu ala cum iti dai seama stabilitatea sistemului dand graficul oribtelor in vector spaceu ala ( sa mai pui si snite poze)
\vskip\bigskipamount

We can observe that the pendulum will not move an inch if it is just left in its "equilibrium" point hanging down ($\omega \equiv \theta \equiv 0$), but that like a coin left on its side, it will also stay still if started at the \textbf{polar} opposite angle in relation to the stability angle ($\omega =0 , \theta = 180 \degree)$, standing up.
Although, like a coin left standing on its side, even the slightest deviation will get it tumbling down. This can be observed in the system's phase portrait.

\newpage

The study of how much such changes in initial conditions (or just the passing of time itself) influence later behaviour is what we call the \textbf{stability}.

In general, for a system: \ref{fo_system_auton_eq}, written in the compact form
\begin{equation}\label{auton_sys_compact}
\dot{y}(t) = \underline{f}(y(t)).
\end{equation}

we have:
\begin{definition}\label{equilibrium_stability}
\textbf{Equilibrium (fixed) points} $y_0$ those satisfying:

$ \underline{f}(y_0) = 0 = \dot{y}(t) |_{y = y_0} $

Which are classified into:

\textbf{(Lyapunov) Stable equilibriums}:
A point $y_0 \in S$ that:

$\qquad \qquad \forall \epsilon > 0, \exists \delta > 0 : $
\begin{equation} \label{cond:stable1}
	\text{\rom{1}: } \forall y(0) \in S \text{ for which } \norm{y(0) - y_0} < \delta \implies \exists y(t), \forall t >= 0.  \tag{\rom{1}}
\end{equation}
\begin{equation} \label{cond:stable2}
	\text{\rom{2}: } \norm{ y(t) - y_0 } < \epsilon \quad \forall t >= 0. \tag{\rom{2}}
\end{equation}

\textbf{Asymptotically stable equilibriums (or attractors)}:
Subset of \textbf{stable equilibria} for which:
\begin{equation}\label{cond:asympt_stable}
	\text{\rom{3}: } \lim_{t \rightarrow \infty} \norm{y(t) - y_0}  = 0 \text{ for those whose } \norm{ y(0) - y_0} < \delta \tag{\rom{3}}
\end{equation}

\textbf{Unstable equilibriums (or repeller)}:
If either \eqref{cond:stable1} or \eqref{cond:stable2} don't hold

And as opposed to, say  a continuum of fixed points we have:

\textbf{Isolated equilibrium}:

A fixed point $y_0$ is said to be \textbf{isolated} if
\[
	\exists \delta > 0 : \forall y \text{ satifying } \norm{y - y_0 } < \delta, \underline{f}(y) \neq 0.
\]
\end{definition}

Classifying the stabilty for a system comes down to looking at the way each state variable evolves with time.

For a linear system, looking at the eigenvalues of the transformation is enough to classify the stabilty of the entire system; taking into consideration how the solutions look; like so:

Taking the eigenvalues
\begin{gather*}
\lambda_1 \dots \lambda_n; \\
\text{ \rom{1} if: } Re(\lambda_i) < 0, \forall i = \overline{1,n} \implies \text{the system is stable }. \\
\text{ \rom{2} if: } \exists I \subset \mathbb{N} \text{ a set of indices } : \forall i \in I, Re(\lambda_i) = 0 \text{ and } \\
\forall k \in \{ 1 , \dots n \} \setminus I, Re(\lambda_k) < 0 \implies  \text{ the system is only \textbf{marginally stable.}} \\
\text{ \rom{3} if: } \exists i = \overline{1,n} : Re(\lambda_i) > 0 \implies \text{system is \textbf{unstable}.}
\end{gather*}

The behaviour can also be classified looking at Im:
\begin{gather*}
Im(\lambda_i) = 0 : \text{The system has no oscillations in the direction of } \lambda_i \text{'s eigenvectors.} \\
Im(\lambda_i) \neq 0 : \text{The system has oscillations in the direction of } \lambda_i \text{'s eigenvectors:} \\
\end{gather*}

This can be seen by \ref{euler} (Euler) because of how complex values are related to rotations.

We can actually plot stabilty diagram for having a visual representation of these classifications:

For a 2x2 Linear system with transform matrix $A$, plotting different systems by their $(Tr(A), det(A))$ plane, where $Tr(A) = \sum_{i}^{n}a_{ii}$ we get:

\begin{figure}[H]
\includegraphics[width=13cm]{math_pics/Stability_Diagram.png}
\centering
\caption{Poincaré Diagram}
\end{figure}

\newpage
Taking the same concept and applying it to non-linear systems we would have:

For a non-linear autonomous system we can look at a zoomed-in portion of whichever equilibrium points we have and we may be able to locally approximate their stabilty by means of the Taylor series at that particular point, which would infer the stabilty of that point in the greater context of the non-linear system.

So for a system \ref{auton_sys_compact}
\[
\dot{y}(t) = \underline{f}(y(t)).
\] that has a fixed point $y_f \implies \underline{f}(y_f) = 0$:

If $f \in C^\infty(\mathbb{R}^n) $, the Taylor about $y_f$:
\begin{equation}\label{eq:t_taylor_expansion}
\underline{f}(y) = \underline{f}(y_f) + \left. \frac{d \underline{f}}{dy} \right|_{y_f}(y - y_f) + \mathfrak{H}(y - y_f) = J_{\underline{f}}(y_f)(y - y_f) + \mathfrak{H}(y - y_f) \tag{Taylor}
\end{equation}

$\mathfrak{H}$ is the rest of Taylor's terms which, for values near the one we're analysing can be neglected since everything looks like a line when you're that zoomed-in eitherway.

$J_{\underline{f}}(y_f)$ is the Jacobian matrix of $\underline{f}$ evaluated at point $y_f$.

\textbf{Reminder:}
\begin{definition}
The Jacobian matrix of a vector-valued vector function $f : D \subseteq \mathbb{R}^n \rightarrow S \subseteq \mathbb{R}^m$, $f(x_1, \dots, x_n) = (y_1, \dots , y_m) \quad [f(x) = (y_1, \dots , y_m)], f \in D(\mathbb{R}^n)$ is $J_f \in \mathbb{R}^{m \bigtimes n}$ an alogue for its first-order derivative, defined as:
\begin{align*}
	J_f = \left[ \frac{\partial f}{\partial x_1}, \dots , \frac{\partial f}{\partial x_n}  \right] =
	\begin{bmatrix*}
		\frac{\partial y_1}{\partial x_1} & \dots & \frac{\partial y_1}{\partial x_n} \\
		\vdots & \ddots & \vdots \\
		\frac{\partial y_m}{\partial x_1} & \dots &  \frac{\partial y_m}{\partial x_n}
	\end{bmatrix*} =
	\frac{df}{dx} =
	\left(   \frac{\partial y_i}{\partial x_j} \right)_{ij | 1 \leqslant i \leqslant n, 1 \leqslant j \leqslant m}
\end{align*}
\end{definition}

\newcommand\eqCuzOfSys{\stackrel{\mathclap{\normalfont\mbox{\ref{auton_sys_compact}}}}{=\joinrel=\joinrel=}}

Using the change of variables $\zeta = y - y_f$ and (\ref{eq:t_taylor_expansion}):
\begin{gather*}
\dot{\zeta} = \dot{y} \eqCuzOfSys \underline{f}(y) \approx J_{\underline{f}}(y_f)(\zeta) \\
\Updownarrow
\end{gather*}
\vspace*{-16mm}
\begin{gather}\label{eq:lin_approx}
\dot{\zeta} \approx J_{\underline{f}}(y_f)(\zeta). \tag{Lin. Approx.}
\end{gather}

To obtain the system's linear estimate at the fixed point $y_f$, which approximates $\underline{f}(y)$ better as $y \rightarrow y_f$.

\begin{theorem}
If $\exists J_{\underline{f}}^{-1}(y_f) \implies y_f$ is an \textbf{isolated eq. point}.
\end{theorem}

\newpage

\textbf{Proof:}
Let's try proving the flip-side of this;

Assume  $\exists J_{\underline{f}}^{-1}(y_f)$ and $y_f$ is not isolated $\implies$

$ \implies \exists y'_f \neq y_f : \underline{f}(y'_f) = 0, \norm{y_f - y'_f} < \epsilon , \forall \epsilon > 0$.

Then, taking (\ref{eq:t_taylor_expansion}) about $y'_f$ and the var. change $\zeta = y - y'_f \implies \dot{\zeta} \approx J_{\underline{f}}(y'_f)(\zeta). $

But because $f(y'_f) =0 \implies J_{\underline{f}}(y'_f)(\zeta) = 0$ and since the substitution has to hold $\forall y : \norm{y - y'_f} < \epsilon$, including $\norm{y - y'_f}|_{y \equiv y_f} > 0 \implies J_{\underline{f}(y'_f)} = 0
\quad \lightning \quad \exists J_{\underline{f}}^{-1}(y'_f)$" \qed

\vspace{20pt}
However, $\exists J_{\underline{f}}^{-1}(y_f) \nLeftarrow y_f$, in general.

Unfortunately, though, using this method we can only find the stabilty of a subclass of equilibrium points and non-linear systems.

So for a nonlinear system $\dot{y} = \underline{f}(y)$ and its (\ref{eq:lin_approx}) $\dot{\zeta} = J(y_f)\zeta$ about fixed state $y_f$ we can only show if (in the context of the non-linear system) that fixed point is either:

1.\textbf{Asymptotically stable} if $Re(\lambda) < 0, \forall \lambda \in Eig(J(y_f))$ (where $Eig(A)$ is the set of all eigenvalues of the matrix A).

2.\textbf{Unstable} if $\exists \lambda \in Eig(J(y_f)) : Re(\lambda)> 0$.

i.e. One can only show Asymptotic (or in-) stabilty, although in the case:

$Re(\lambda) \leqslant  0 \forall \lambda \in Eig(J(y_f)) \text{ and,
at least } \exists \lambda : Re(\lambda) = 0$ then the results are
inconclusive and can't be implied for the non-linear system.

\hfill\break
//TODO: HAI MA CA NU ESTI PROST invata si despre lyapuov mai bine sa

poti sa vorbesti despre dansul dupa ce zici din astea mai de
specifice de chimie ca sigur o sa ajute cu ceva na pana mea

5. arata poza de clasificarea sistemelor, dat

\hfill\break
But another way of analysing the stabilty would be by using one of
the most fundamental tools during our study, so we'll define and show
how to obtain:

\textbf{The Routh-Hurwitz matrix and its determinants.}

The english mathematician Edward John Routh was looking at the
(\ref{eq:lin_approx}) of
\[
x_i' = f_i(x_1,\dots , x_n), \quad i = \overline{1,n}
\]
Namely:
\[
x_i' =\sum_{j=1}^{n}a_{ij} x_j, \quad a_{ij} = \frac{\partial
f_i}{\partial x_j}(0)
\]
He figured out he doesn't have to find the roots of
\begin{equation}\label{lin_apporx_char_equation}
det(J - \lambda I_n) = a_0 \lambda^n + a_1 \lambda^{n-1} + \dots +
a_{n-1}\lambda + a_n = 0
\end{equation}
To infer the system's stability, as in check
$ Re(\lambda_0) < 0$,

but can instead instead just simply look at the coefficients of
\ref{lin_apporx_char_equation}: $a_{i}$ to verify it.

To do this he used:

\rom{1}: Cauchy's argument principle

and

\rom{2}: Sturm's theorem

\rom{1}

Cauchy's argument principle states that for the poly. $p \in \mathbb{C}[Z]$:
\[
p(z) = u(z) + i v(z), \quad u, v \in \mathbb{R}[Z]; u,v : \mathbb{C}
\rightarrow \mathbb{R}
\]
Its number of roots inside a closed contour $\gamma$  is $=$ to the
number of positive rotations (counter-clockwise)[c.c.] that $(u(z),
v(z))$ (or $arg(p(z))$) makes as $z$ moves across $\partial \gamma$
in a positive sense [c.c.].

\hfill\break
//TODO: ai putea sa pui si tu o poza cu un vector space format de un
\hfill\break

anumit polinom complex cu valori complexe

Now all we need to do is make this "closed contour" the entire
left-half of the complex axis!

We could write it as a {\large HUGE} half-circle
\[
z = Re^{i \theta}, \quad \frac{\pi}{2} \leq \theta \leq \frac{3
\pi}{2}, R \text{ sufficiently large }
\]

The thing is even if it's the case that all $n$ roots lie inside it,
$arg(p(z))$ will only make $\frac{n}{2}$ rotations since this is a
half circle, so we could combine this fact with a more elegant way of
writing "the half-plane, left of the imaginary axis" into:

\begin{lemma}\label{cauchy_arg_lemma}
For $p(z) \in \mathbb{C}_n[Z]$ and $p(ix) \neq 0. \forall x \in
\mathbb{R}$. Then $\forall z_0 : p(z_0) = 0, Re(z_0) < 0 \iff$
$arg(p(ix))$ makes $\frac{n}{2}$ positive rotations for $x \in
[-\infty , \infty]$.
\end{lemma}

\rom{2}.

The second tool he's used was Sturm's theorem, which was the result
of his discovery that for the division $\frac{p_{i-1}(x)}{p_i(x)}$,
for $p_k \in \mathbb{R}[X]$ it's better to take the remainder
$p_{i+1}(x)$ with a negative sign:
\begin{equation}\label{euclid_algorithm}
p_{i-1}(x) = p_i(x) q_i(x) - p_{i+1}(x) \tag{Euclid's Algorithm}
\end{equation}
Because, if we do that, then:
\begin{equation}\label{sturm_seq_property}
\text{sign} ( p_{i+1}(x) )  \neq \text{sign} ( p_{i-1}(x) ), \quad
\text{for} \quad p_i(x)=0 \tag{Sturm. Seq. Property}
\end{equation}
This is called the "Sturm sequence property"

And if we construct a sequence
\[
(p_k)_{k = \overline{0,m}} \subset \mathbb{R}[X]
\]
and the function
\begin{gather}\label{no_sign_changes_func}
w : \mathbb{R} \rightarrow \mathbb{R} \notag  \\
w(x) = \text{No. of sign changes of  } (p_k)(x) \notag
\end{gather}
Then we find that for:
\begin{gather*}
p_i^{-1}(0) \text{ ( set of roots for $p_i$) } \\
R := \bigcup_{i \in \left\{ 1, \dots, m-1 \right\} } p_i^{-1}(0) \\
\forall x_1, x_2 \in R, w(x_1) = w(x_2)
\end{gather*}
Hence we have:
\begin{lemma}\label{sturm_lemma}
For a seq. of $  (p_k)_{k = \overline{0,m}} \subset \mathbb{R}[X]$
that satifies:

\rom{1}
$ deg(p_0) > deg(p_1), $

\rom{2}
$\nexists x \in \mathbb{R} : p_0(x) = p_1(x) = 0, $

\rom{3}
$p_m(x) \neq 0, \forall x \in \mathbb{R}$

\rom{4}
And (\ref{sturm_seq_property})
\par
We have (for the above defined $w$ \ref{no_sign_changes_func}) :
\begin{gather}
	\frac{w(\infty) - w(-\infty)}{2} =  \notag \\
	\text{No. of positive rotations of the vector } \notag \\
	(p_0(x), p_1(x)), \text{as $x$ tends from  } x \in [-\infty, +\infty] \notag
\end{gather}
\end{lemma}

Proof: As stated in \ref{sturm_seq_property}: $w(x)$ doesn't change at roots of

$p_k(x), k = \overline{1,m-1}$. Asumption \rom{3} shows $p_m(x)$
doesn't influence it either. So there can only be one suspect:
$p_0(x)$. If there's a spike by $1$ at $\overline{x}$ in $w(\overline{x})$,

zero of $p_0$
(   $p_0(\overline{x}) =0 $) either:
\[
p_0(x) \text{ changes from + to - and   } p_1(\overline{x}) > 0
\]
or:
\[
p_0(x) \text{ changes from - to + and  } p_1(\overline{x}) < 0
\]
[   $\nexists \overline{x}: p_1(\overline{x}) = 0$ by assumption \rom{2}  ]

Both of which cause the vector $(p_0(x), p_1(x))$ to cross the
imaginary axis in a positive sense.

If there's a dip in $w$ at $\overline{x}$, however, the crossing is
done clock-wise, this time.

Now taking \rom{1} into account and the fact that $(p_0(y), p_1(y))$
is horizontal for $y \rightarrow  - \infty$ and $y \rightarrow +
\infty  $ outcomes the result of our lemma
\qed

\hfill\break
//TODO Maybe try to draw this yourself cu arrows nstuff
\hfill\break

\begin{figure}
\includegraphics[width=13cm]{math_pics/sageti-cumse-invart.png}
\centering
\end{figure}

So using these 2 Lemmas we have the criteria for stabilty:
For the characteristic polynomial \ref{lin_apporx_char_equation},
written instead like this:
\[
p(z)=a_0 z^n + a_1z^{n-1} + \dots + a_n = 0, \quad a_0 > 0
\]

Dividing $\frac{p(i x)}{i^n}$ and using \ref{real_imag_another_way_to_write}
\begin{gather}
p_0(x)= \text{Re}(\frac{p(i x)}{i^n}) = a_0 x^{n} -a_2 x^{n-2} + a_4
x^{n-4} \pm \dots \\
p_1(x)= -\text{Im} (\frac{p(i x)}{i^n}) = a_1 x^{n-1} -a_3 x^{n-3} +
a_5 x^{n-5} \pm \dots
\end{gather}

More generally:
\begin{equation}\label{polys_general_form}
p_i(x) = c_{i0} x^{n-i} + c_{i1} x^{n-i-2} + c_{ i2 } x^{ n-i-4  } +
\dots, \tag{Gen. Form}
\end{equation}

And $q_i(x) = (  \frac{c_{ i-1 ,0}}{c_{ i0 }} )x$ from
\ref{euclid_algorithm} given that $c_{i0} \neq 0$.
Putting \ref{polys_general_form} into \ref{euclid_algorithm} to get
the general form of all the coefficients as well we get:
\begin{align}\label{coeff_gen_form}
c_{i+1,j} = c_{i,j+1} \cdot \frac{c_{i-1,0}}{c_{i,0}} - c_{i-1,j+1 }
= \frac{1}{c_{i,0}} \text{det}
\begin{pmatrix}
	c_{i-1,0} & c_{i-1, j+1}            \\
	c_{i,0}   & c_{i, j+1} \tag{Coeff.}
\end{pmatrix}
\end{align}

The algo. stops for $ p_m(x) $ with $ m < n$ if $c_{i,0} = 0$ at a
particular $i \implies q_i(x)$ is of higher degree.

And so: $(p_i(x))_{i = \overline{1,m \text{ or } n}}$ verifies
\rom{1}, \rom{4} from \ref{sturm_lemma}

\rom{2} $\iff p(ix) \neq 0, \forall x \in \mathbb{R}$

$p_m(x) = \text{gcd}(p_0(x), p_1(x)) \iff \text{ \rom{2} } \implies
\text{ \rom{3} }$

\par

So here's the big guy:

\begin{theorem}\label{routh_theorem}
Routh:
\begin{gather*}
	Re(\lambda) < 0, \quad \forall \lambda \in \text{det}(J - \lambda
	I_n)^{-1}(0) \text{ from \ref{lin_apporx_char_equation} with } a_0 > 0 \\
	\Updownarrow  \\
	c_{i,0} > 0, \forall  i = \overline{1,n} \text{ from \ref{coeff_gen_form} }
\end{gather*}
\end{theorem}
Meaning the system is stable.

\textbf{Proof: } $arg(p_0,p_1) = 360^\degree - arg(\text{Re}(p),
\text{Im}(p)) \implies \frac{n}{2}$ positive rotations of $p(ix)$ are
bascially $\frac{n}{2}$ neg. rot. of $(p_0(x), p_1(x))$. If
$Re(\lambda) < 0, \forall \lambda \in p(\lambda)^{-1}(0) \implies$ (
from  \ref{cauchy_arg_lemma}, \ref{sturm_lemma} ) $w(\infty) - w( -
\infty) = -n$. Could only be for $w(\infty) = 0, w(- \infty) = n$,
though. $ \implies p_i(x) \geq 0$ for all leading $i$. Having
\ref{routh_theorem} $\implies p_n(x) \equiv c_{n0}$. Because
$\nexists \text{gcd}(p_0(x),p_1(x)), p(\lambda) \neq 0$ for
Re$(\lambda) = 0$. Now we just do \ref{cauchy_arg_lemma} +
\ref{sturm_lemma} once more. \qed

\hfill\break
//TODO vezi ba sa rezolbi asta cumva ca nu esti chiar asa prost te pui cu pixu cu astea sa vezi de unde vine si faci pasii de rezolvare de le-a facut hurwitz intr-un fel
\hfill\break

\includegraphics[width=13cm]{math_pics/TODO SA FACI KKTU ASTA BA.png}

Adolf Hurwitz later wrote the positivity criterion ($2^{nd}$ condition in \ref{routh_theorem}) making use of this matrix we now basically call "his".
\begin{theorem}
Hurwitz:
\begin{align*}
	H =
	\begin{pmatrix*}
		a_{2j-i}
	\end{pmatrix*}^n_{i,j=1} =
	\begin{pmatrix*}
		a_1 & a_0 & 0 & 0 & 0 & \dots & 0 \\
		a_3 & a_2 &  a_1 & a_0 & 0 & \dots &  0 \\
		\vdots & \vdots & \vdots & \vdots & \vdots & \ddots & \vdots \\
		a_{ 2n-1 } &  a_{ 2n-2 } & a_{ 2n-3 }  & a_{ 2n-4 }  & a_{ 2n-5 }  & \dots & a_n
	\end{pmatrix*}
\end{align*}

\end{theorem}

\vskip\bigskipamount

\chapter{Existence and Absence of Hopf Bifurcation in Phosphorylation-Dephosphorylation CRN}

\section{What even \textbf{is} a bifurcation?}
To better generalize them, it'd be best to also define the notion of:
\begin{definition}	
	\textbf{Invariant sets}. \footcite{faye2011Bifurcation} \footcite{afrajmovich1999Bifurcation}

	Taking the Auton. sys.
	\begin{equation}\label{invar_set_auton_sys}
		\dot{y} = f(y) , y(0) = y_0, y \in \mathbb{R}^n
	\end{equation}

	A state-set $S \subseteq \mathbb{R}^n$ of \ref{invar_set_auton_sys} is \textbf{invariant} if $\forall y_0 \in S, \forall t \geq 0: y(t) \in S$
\end{definition}

As you can see by this above definition, equilibria are also particular cases of invariant sets, themselves.

They also fall into stability classes that can be defined as in Def. \ref{equilibrium_stability}, but replacing
\[
	"y_0" \text{ with } "S \subseteq \mathbb{R}^n \text{ }"
\]
and
\[
	" \dotso \norm{y(t) - y_0} \dotso" \text{ with } "\dotso \text{ inf}\{ \norm{y(t) - s} : s \in S  \} \dotso \text{ }"
\]
The Jacobian (linearization), generally, would also not be as useful in this context anymore.
\newpage
Now consider we have a system as above, except it now depends on an extra parameter $\mu$, which remains constant \textbf{during} our "experiment".
\begin{equation}\label{auton_parameter_sys_compact}
	\dot{y} = f_\mu(y).
\end{equation}

This could be, for example, the length of the pendulum $L$, in \ref{damped_pendulum}, or its damping coefficient $c$. Honestly generalizing it even more, it could even be the gravitational constant $g$. Hence, this parameter can, of course be a \textbf{vector} of such parameters.

The way the parameter $\mu$ varies also induces variations in the topology and dynamics of the system. That's obvious since this is kind of the point of parameters in systems.

What can vary though can be trivial or more interesting.

The positions of invariant sets, for example can vary continuously

during changes in $\mu$; that's usually normal.

But what's more interesting is when these change their \textbf{stability} altogether, equilibria disappear completely / appear out of thin-air - or, even - collide with other invariant sets.

This odd behavior is called a \textbf{bifurcation}, and $\mu$, in this case is a \textbf{bifurcation parameter}.

The former types of bifurcations I talked about previously are called \textbf{local}, those in which changes in stability can only be considered mostly for \textbf{isolated equilibria}, individually, or other invariant sets whose stability can be analyzed \textbf{locally}, via the system's linearization at a point.

The latter are \textbf{global}, which are more complex and require greater understanding of the system beyond a local linearization, but that falls into the special-case category described in the last paragraphs of the previous chapter, and so far all I know about Mr. Prof. Lyapunov as of now is that his brother was a pianist. But thankfully our interest for this work lies with local bifurcations, so we'll be ignoring this for now, maybe I'll come back to it during my Master's (not).

To better define and illustrate them for the 1-D case and give example of bifurcations for such dimension, we could instead write our vector field $f$ as depending on $\mu$ in addition to the unknown function $y$.
\begin{equation}\label{eq:1-d_bif_sys}
	\dot{y} = f(y, \mu).
\end{equation}
And assume $f \in C^k(\mathbb{R} \bigtimes \mathbb{R}), k \geq 2 $ around $(0,0)$, and
\begin{equation}\label{bifurcation_priming}
	f(0,0) = 0, \quad \frac{\partial f}{\partial y}(0,0) = 0.
\end{equation}
By these conditions we are basically "priming" the system for bifurcations, but none are enough for them to actually occur. Here come the specific definitions for each particular type of bifurcation in 1-D.

\begin{definition}
	A \textbf{Saddle-node bifurcation}:
	for $f$ in \ref{bifurcation_priming}, add as well:
	\begin{equation*}
		\frac{\partial f}{\partial \mu}(0,0) =: a \neq 0, \quad \frac{\partial^2 f}{\partial y^2}(0,0) =:b \neq 0.
	\end{equation*}
\end{definition}
Then, for \ref{eq:1-d_bif_sys}, a saddle-node bifurcation occurs at $\mu = 0$, if the following equivalences are true:

\rom{1}. For $ab < 0$ (resp. $ab> 0$), $\nexists y : f(y,\mu) = 0$ for $\mu < 0$ (resp. $\mu > 0$)

\rom{2}. For $ab < 0$ (resp. $ab > 0$), $\exists y_+(\epsilon) \neq y_-(\epsilon) \implies y_\pm(\epsilon) : f(y_\pm(\epsilon), \mu) = 0, \epsilon = \sqrt{\abs{\mu}}$, for $ \mu > 0 $ (resp. $\mu < 0$), having opposing stabilities.

\begin{definition} \label{pitchfork_bif}
	\textbf{Pitchfork bifurcation}.

	For $f$ from \ref{bifurcation_priming}, assume as well that:
	$f \in C^k, k \geq 3$,
	\[
		f(-y, \mu) = -f(y,\mu)
	\]
	and,
	\[
		\frac{\partial^2 f}{\partial \mu \partial y}(0,0) =: a \neq 0, \quad \frac{\partial^3 f }{\partial y^3}(0,0)=: b \neq 0
	\]

	Now, a \textbf{pitchfork bifurcation} occurs for \ref{eq:1-d_bif_sys} at $\mu  =0$ if:

	\rom{1}. for $ab < 0$ (resp. $ab > 0$) $\exists! y = 0 :f(y,\mu) = 0 $ for $\mu < 0$ (resp. $\mu > 0$). $b < 0 \implies$ stable, $b > 0 \implies$ unstable.

	\rom{2}. for $ab < 0$ (resp. $ab > 0$) $\exists y = 0 : f(y , \mu) = 0$, as well as: $y_+(\epsilon) \neq y_-(\epsilon) \implies y_\pm(\epsilon) : f(y_\pm(\epsilon), \mu) = 0, \epsilon = \sqrt{\abs{\mu}}$  for $\mu > 0$ (resp. $\mu < 0$), for which $y_+(\epsilon) = -y_-(\epsilon)$, Both $y_-(\epsilon)$ and $y_+(\epsilon)$ have matching stabilities, whereas $y = 0$ has the opposite stability of them, given by $b < 0 \implies$ stable, $b > 0 \implies$ unstable.
\end{definition}

\begin{definition}
	\textbf{Transcritical bifurcation}.

	For $f$ in \ref{bifurcation_priming}:
	\[
		\frac{\partial^2 f}{\partial \mu \partial y}(0, 0) =:a \neq 0, \quad \frac{\partial^2 f}{\partial y^2}(0,0) =: b \neq 0
	\]
	Then a \textbf{transcritical bifurcation} occurs for \ref{eq:1-d_bif_sys} at $\mu = 0$, char. by (if they hold):

	\rom{1}.  $\exists y = 0 : f(y, \mu) = 0$  and  $ \exists y_0(\mu) : f(y_0(\mu), \mu) = 0 $ where $\mu \rightarrow u_0(\mu) \in C^m, m = k - 2$

	\rom{2}.  for $a\mu  < 0$ (resp. $a\mu > 0$) the trivial fixed point $y = 0 \implies$ stable (resp. unstable)
	and $u_0(\mu)$ has opposing stability.
\end{definition}

Finding local bifurcations can be done by looking at eigenvalues.
\newpage
{\large The local bifurcation condition

}For system:
\begin{equation}
	\dot{y} = f(y, \mu) \quad f : \mathbb{R}^n \times \mathbb{R} \rightarrow \mathbb{R}^n
\end{equation}

\begin{definition}\label{local_bif_def}
	$(y_0,\mu_0)$ admits a \textbf{local bifurcation} if:
\end{definition}

If $\forall \lambda \in Eig(J_{f}(y_0,\mu_0)) : Re(\lambda) \leq 0$, while $\exists \lambda_0 : Re(\lambda_0) = 0$

\begin{equation}\label{stable_state_bif}
	\text{If } \lambda_0 = 0 \text{ the bifurcation is stable-state defined. }
\end{equation}

We also have a second kind of local bifurcation and this is where we begin the crux of the thesis, and answer the first question I had getting into this:

\section{What is a simple Hopf Bifurcation?}

A \textit{simple Hopf bifurcation} is a bifurcation in which a single complex-conjugate pair of eigenvalues of the Jacobian matrix crosses the imaginary axis, while all other eigenvalues remain with negative real parts. Such a bifurcation generates nearby oscillations or periodic orbits.

Or, more formally, defined similarly to Def. \ref{local_bif_def}:
\begin{definition}
	$(y_0, \mu_0)$ admits a \textbf{Hopf bifurcation} if we have Def. \ref{local_bif_def}, and additionally:
	\begin{equation}\label{hopf_bif_def}
		\exists! \lambda_0 : Re(\lambda_0) = 0, \quad
		Im(\lambda_0) \neq 0.
	\end{equation}
\end{definition}
Visually, just to hammer it in:

\begin{figure}[H]
	\includegraphics[width=13cm]{math_pics/hopf-bif-eigenvalue-graph.png}
	\centering
	\caption{Visualizing the zeroes \footcite{vogel2020HopfEigenvalues}}
\end{figure}

In $\mathbb{R}^2$, such Bifurcations cause special kinds of invariant sets, namely \textbf{limit cycles} to appear. These are periodic solutions, for which we'll need a few more concepts in order to be able to define them:
\begin{definition} \textbf{Closed trajectory and Closed orbit (cycle)}

	A trajectory (or flow): $y(t) \in \mathbb{R}^2$ as in Def. \ref{dyn_sys_orbit_flow_etc_def} is a solution for \ref{eq:1-d_bif_sys}

	A closed trajectory is one which returns to its starting point $\forall s_0 := \Phi(0,s) $ it has, namely:
	\begin{gather*}
		\exists t_0 \in T : \Phi(t,s) = \Phi(t+t_0,s), \forall s \in \Phi_{s_0}     \\
		\Updownarrow \\
		\exists t_0 \in T : y(t) = y(t+t_0), \forall t \in T
	\end{gather*}
	A \textbf{closed orbit} or \textbf{cycle} is just the image of such a closed trajectory.
\end{definition}

\begin{definition}\textbf{Limit point}
	These can be grouped into $\omega$ (attracting) and $\alpha$ (repelling) limit points:

	$y_\omega$ is an $\omega$-limit point  for $y$ if:

	$\exists (t_n)_{n \in \mathbb{N}} \subseteq I(y) : $
	\begin{gather*}
		\lim_{n \rightarrow \infty} t_n = \infty  \\
		\lim_{n \rightarrow \infty} \Phi(t_n,y) = y_\omega
	\end{gather*}
	And similarly, but in reverse:

	$y_\alpha$ is an $\alpha$-limit point  for $y$ if:

	$\exists (t_n)_{n \in \mathbb{N}} \subseteq I(y) : $
	\begin{gather*}
		\lim_{n \rightarrow \infty} t_n = \textbf{--} \infty  \\
		\lim_{n \rightarrow \infty} \Phi(t_n,y) = y_\alpha
	\end{gather*}
\end{definition}

\begin{definition}\textbf{Limit set}
	The set of all $\omega$ (or $\alpha$)-limit points for a particular orbit $\gamma$ is called the \textbf{limit set} of $\gamma$, denoted and defined as:
	\[
		\lim_{\omega}\gamma_s := \bigcap_{t \in T} \overline{ \{ \Phi(t', s) : t' > t \} }
	\]
	\[
		\lim_{\alpha}\gamma_s := \bigcap_{t \in T} \overline{ \{ \Phi(t', s) : t' < t \} }
	\]
\end{definition}

\begin{definition} \textbf{Limit cycle}(finally)

	A Limit Cycle is a cycle which is the limit set of at least another trajectory.
	Also, interestingly:
	\[
		\lim_{\omega}\gamma \bigcap \gamma  = \emptyset \implies \text{ it's an } \omega \text{-limit cycle}.
	\]
	\[
		\lim_{\alpha}\gamma \bigcap \gamma  = \emptyset \implies \text{ it's an } \alpha \text{-limit cycle}.
	\]
\end{definition}

What's interesting though, and makes calculations easier for the 2-D case is:
\begin{theorem}  \textbf{Poincaré-Bendixson}

	For a dynamical system $(T,X,\Phi)$ with $X \subseteq \mathbb{R}^2$, $\forall$ compact invariant set $S$:
	\begin{gather*}
		\text{if } \nexists x_0 \in S : \Phi(t_1, x_0) = \Phi(t_2,x_0), \forall t_1,t_2 \in T \\
		\Downarrow \\
		\forall s \in S: \gamma_s \text{ are either limit cycles, or } \lim_{\omega / \alpha}\gamma_s \text{ is an $\omega$ (or $\alpha$)-limit cycle }.
	\end{gather*}
\end{theorem}

But the problem is this theorem only holds for $\mathbb{R}^2$. For higher dimensions we don't have this property necessarily, instead we have to look for periodic solutions (limit cycles) ourselves - which, as stated previously - occur naturally during a Hopf bifurcation.

So having these in mind, Hopf is kind of like an $\mathbb{R}^n$ version of Pitchfork bif.: \ref{pitchfork_bif} when you think about it.

You can even see the resemblance in the way they look \footcite{brainsci10080536}

\begin{figure}[H]
	\includegraphics[width=13cm]{math_pics/hopf-bif-pic.png}
	\centering
\end{figure}

\begin{figure}[H]
	\includegraphics[width=13cm]{math_pics/pitchfork-photo.png}
	\centering
	\caption{Pitchfork bifurcation \cite{Yang2020}}
\end{figure}

\subsection{Proving the existence of simple Hopf bifurcations}

Finding necessary conditions for their existence makes use of a tool defined in the previous chapter (\ref{hurwitz_theorem}) the \textbf{Hurwitz matrix}. \footcite{LIU1994250}

Instead, now the characteristic polynomial depends on this new bifurcation parameter $\mu$.

\begin{proposition}\label{symmetric_roots_pair_criterion}
	Take $p_{\mu_0} \in \mathbb{C}^n[Z]$ with $n \geq 2$ and consider $\mu_0$ fixed:
	\[
		p_{\mu_0}(z) = a_0(\mu_0)z^n + a_1(\mu_0)z^{ n-1 } + \dots + a_n (\mu_0)
	\]

	Take $H_i(\mu_0)$ to be $p$'s $i$-th Hurwitz Matrix (\ref{hurwitz_theorem})

	If we have:
	\begin{gather*}
		\det H_1(\mu_0) > 0 , \dots, \det H_{n-2} (\mu_0) > 0. \\
		\Downarrow \\
		\text{ pair of symmetric roots of   } p_{\mu_0}(z) \leq 1
	\end{gather*}

	And $\exists!$ pair of symmetric roots $\iff \det H_{n-1}(\mu_0) = 0$.

	If $a_n(\mu_0) > 0 \implies$ for the pair $\lambda_i, \overline{\lambda_i}$  : Re$(\lambda_i) =$ Re$(\overline{\lambda_i}) = 0$.
\end{proposition}

Now, more specifically, the necessary conditions for a simple Hopf bifurcation are:

\begin{proposition}\label{yangs_criterion}
	\cite{Yang2002}
	A simple Hopf bifurcation occurs at fixed point $x^*$ and at the parameter threshold $\mu_0$ $\iff$
	\begin{gather*}
		\rom{1} \quad \det  H_{ n-1 }(\mu_0) = 0 \text{  and  } a_s(\mu_0) > 0, \\
		\rom{2} \quad 		\det H_1(\mu_0) > 0 , \dots, \det H_{n-2} (\mu_0) > 0 \text{  and  } \\
		\rom{3} \quad \left. \frac{d(\det H_{n-1}(\mu))}{d\mu} \right|_{\mu = \mu_0} \neq 0.
	\end{gather*}
\end{proposition}

\subsection{Ruling out simple Hopf bifurcations}

Directly from criterion \ref{symmetric_roots_pair_criterion} we have a criterion for showing $\nexists \mu$ for which $x(\mu)$ undergoes simple Hopf bifurcations.
\begin{theorem}\label{ruling_out_simple_hopf_bif}

	For the dynamical system $\dot{x} = f_\mu(x)$ assume $\exists$ a curve of steady states $x(\mu)$.
	$p_\mu(z)$ is the characteristic polynomial of degree $s \geq 2$ of the linearization $J(x(\mu), \mu)$, and $H_i(\mu)$ its $i$-th Hurwitz matrix. Given:
	\[
		\det H_1(\mu_0) > 0 , \dots, \det H_{n-2} (\mu_0) > 0. \forall \mu
	\]
	And either:
	\begin{gather*}
		a_s(\mu) \leq 0 \text{ whenever  } \det H_{s-1} = 0  \\
		\text{or} \\
		\det   H_{s-1}(\mu) \neq 0, \forall \mu    \\
		\Downarrow
	\end{gather*}
	$\nexists$ simple Hopf bifurcation $\forall \mu$ at the steady states $x(\mu)$.
\end{theorem}

\subsection{Convex parameters}\label{convex_paramteres}
\footcite{ErramiEtAl2015a} \footcite{Rockafellar1973}
With most dynamical systems, if we want to analyze their behavior we have to find, through all possible values if $\exists x^*(\mu)$ fixed points for which bifurcation occur, and hence we need to find any value $\exists \mu$ of the bifurcation parameter for which said bifurcation occurs, throughout our entire domain.

But luck would have it, though that the kind of systems we care about in this thesis, namely chemical reaction systems, defined in more detail in \ref{mass-action_network}, are not like "most" dynamical systems.

They all have a certain structure and their corresponding differential equations look in a way that can be represented more manageably.

Take now a system for a CRN, written as well in its matrix form, as in \ref{crn_system_matrix_form}
\[
	\dot{x} = f(k,x) :=\Gamma v(k,x)
\]
Where $v$, the flux vector, as in \ref{flux_vector} can be written as:
\[
	v(k, x) = \text{diag}(k)\Psi(x).
\]
as well.
Where diag $: \mathbb{C}^n \rightarrow \mathcal{M}_n(\mathbb{C})$,
\begin{align*}
	(\text{diag}(x))_{i,i} = x_i, \quad \forall i = \overline{1,n} \\
	(\text{diag}(x))_{i,j} = 0, \quad \forall i,j = \overline{1,n} , i \neq j.
\end{align*}
And $\Psi(x)$ is the flux vector, where the reaction rates \ref{reaction_rate} are written without the leading $k_i$'s, only depending on $x$.

If we want to look into how the system behaves near equilibrium points $x^*$ we have to, just like other systems, study their Jacobian matrix, which, because the flux vector is made of polynomials can be written using this little trick:
\[
	J(k, x)_{\mid x=x^*}=\Gamma \operatorname{diag}\left(v\left(k, x^*\right)\right) \Gamma_L^T \operatorname{diag}\left(\frac{1}{x^*}\right)
\]

Now since $x^*$ is a fixed point, the flux vector satisfies:
\begin{equation}\label{flux_vector_linear_problem}
	\Gamma v(k,x^*) = 0, \quad v \geq 0.
\end{equation}
And so the solutions of (\ref{flux_vector_linear_problem}) (ker$\Gamma$) are vectors (rays) which describe a \textbf{convex polyhedral cone} \footcite{dattorro2018Convex} called the \textbf{flux cone}.

It's been shown in (\cite{clarke1980stability}) that:

$(k,x^*)$ satisfies \ref{flux_vector_linear_problem} $\iff  v(k,x^*)\in \text{ker}(\Gamma) \bigcap \mathbb{R}^r_{\geq 0}$

Where $r:$ no. of reactions.

A flux cone is expressed as an $\mathbb{R}_{\geq 0}$ linear combination (positive hull) of its extreme vectors $\left\{ E_1 , \ldots , E_l \right\}$. Also, we need $E_i \neq 0_r, \forall i = \overline{1,l}$

\begin{definition}\label{convex_params_definition}
	\textbf{Convex parameters.}
	\begin{equation}\label{flux_cone}
		\boxed{		v=\sum_{i=1}^l \lambda_i E_i=E \lambda, \quad \lambda_i \geq 0, \forall i = \overline{1,l} }
	\end{equation}
	Where $\lambda_i$ are called \textbf{convex parameters}.
\end{definition}
But we also need something to parametrize the $x$'s, so:
\begin{equation}\label{other_convex_parameters}
	\boxed{	h_i=\frac{1}{x_i^*}, \quad i = \overline{1,n} }
\end{equation}
So having this in mind:
\begin{definition}
	A convex parameter vector:
	\[
		(h, \lambda) = (h_1, \ldots, h_n , \lambda_1, \ldots , \lambda_l) \in \mathbb{R}_{>0}^n \times \mathbb{R}_{\geq 0}^{l} : E \lambda \in \mathbb{R}^r_{> 0}.
	\]
	With $E$ the matrix whose columns are the convex polyhedral cone's extreme vectors.
\end{definition}
So we may write the Jacobian using this new coordinate system, which in turn makes its coefficients monomials, instead of the usual polynomial and multiple-term expressions you usually get with these systems, like in Ex. (\ref{bigger_network_example1}).
\newcommand\eqCuzConvex{\stackrel{\mathclap{\normalfont\mbox{\ref{flux_vector_linear_problem}, \ref{convex_params_definition}}}}{=\joinrel=\joinrel=\joinrel=\joinrel=}}

\begin{gather}\label{jacobian_convex_params}
	v(k, x^*) \eqCuzConvex E \lambda  \notag \\
	\Downarrow \text{ \ref{other_convex_parameters} } \notag \\
	\boxed{J(k,x)_{|x=x^{*}}=J(h,\lambda)=\Gamma \text{diag}(E\lambda)\Gamma_{L}^{T}\text{diag}(h)}
\end{gather}

We could, of course turn this convex vector back into the regular coordinate system:

Given:
\begin{gather}\label{k_convert_back_from_convex}
	(h,\lambda) \Downarrow \notag \\
	\text{Let  } x^* \in \mathbb{R}^n , \boxed{
		x^*_i = \frac{1}{h_i} , i = \overline{1,n}
	} \notag \\
	\boxed{
		k = \text{diag}(\Psi(h)) E \lambda \in \mathbb{R}^r_{> 0}
 }
\end{gather}

There is one particular case that is of interest to us, though.

\section{What is a Phosphorylation-Dephosphorylation CRN?}

Phosphorylation of proteins occurs in cycles, which are fueled by $3$ proteins which are the ingredients: a substrate $S$ and $2$ enzymes: kinase ($K$) and phosphatase ($F$).

The one that starts this chain is the kinase ($K$), attaching phosphate groups onto the substrate, phosphorylating it.

Then, phosphatase ($F$) comes in and undoes all the hard work kinase ($K$) put in and removes the phosphate groups, now dephosphorylating the substrate.

\hfill\break
These cycles are a particular case of a broader type of what are called \textbf{posttranslational modification (PTM) systems.} \footcite{conradi2024} Another example would be, for instance \textbf{methylation}, where in this case methyl groups are the ones attaching to specific sites. \footcite{ramazi2021Ptms, SCHOENHEIMER1939333}

The reason we care about them is their key implication in \textbf{ signal transduction}, which is the process our cells use to communicate with one-another. Any disturbance in this system is linked to its own class of health complications in our body. \footcite{CONRADI2018507, 10.1093/hmg/ddp186, cohen2001Phosphorylation}

Basically, do these biochemical systems induce periodic solutions as in (\ref{hopf_bif_def}), likening clocks, or do they have multiple steady-states (\textbf{capacity for bistability}), likening switches?

What these all have in common, though are the \textbf{building blocks} used to create them:
\begin{equation}\label{ptm_building_blocks}
	S+M \xrightleftharpoons[k_2]{k_1} SM \xrightarrow{k_3} S^* + M \quad \mathrm{~and~} \quad S^* + U \xrightleftharpoons[k_5]{k_4} S^* U \xrightarrow{k_6} S + U
\end{equation}

What's familiar to the previous example is the presence of a substrate ($S$), which forms the \textbf{complex} ($SM$) with the \textbf{modifier} ($M$), which - as the name implies - modifies ($S$) to become ($S^*$), which then dissociates with ($M$);

And just like in dephosphorylation, another modifier ($U$) can come in and undo the entire hustle ($M$) did.

So the \textbf{phosphorylation version of this would be}:
\begin{gather*}\label{phosphorylation_reaction_basis}
	S_0 + K \xrightleftharpoons[k_2]{k_1} K S_0 \xrightarrow{k_3} S_1 + K  \xrightleftharpoons[k_5]{k_4} \ldots K S_{n-1} \xrightarrow{k_{ 3n}} S_n + K  \\
	\text{ for the phosphorylation side, and} \\
	S_n + F \xrightleftharpoons[k_{2(n+1)}]{k_{2n+1}} F S_n \xrightarrow{k_{2n + 3}} S_{n-1} + F  \xrightleftharpoons[k_{2n + 5}]{k_{2n + 4}} \ldots F S_{1} \xrightarrow{k_{6n}} S_0 + F  \\
	\text{ for the dephosphorylation side, and}
\end{gather*}
Which phosphorylates the substrate ($S$) up to a certain number of phosphate groups $n$, and then \textbf{de}phosphorylates it back to 0, where the subscript notation $S_i ;  i = \overline{1,n}$ represents the substrate ($S$) with its $i$ phosphate groups.

\subsection{Cyclic and mixed distributive and processive Phosphorylation-Dephosphorylation CRN}
The system above represents a \textbf{distributive} cycle, meaning each bounding site of ($K$) and ($F$) (de)phosphorylate only one single phosphate group.

As opposed to a \textbf{processive} one, which would look something like:
\begin{equation*}
	S_0 + K \xrightleftharpoons[k_2]{k_1} S_0 K \xrightarrow{k_3} S_1 K \xrightarrow{k_4} S_2 + K
\end{equation*}
For a binding one, and
\begin{equation*}
	S_2 + F \xrightleftharpoons[k_2]{k_1} S_2 F \xrightarrow{k_3} S_1 F \xrightarrow{k_4} S_0 + F
\end{equation*}
For the unbinding operation.

Notice these extra reactions where the enzymes don't dissociate.
\begin{gather*}
	\ldots S_0 K \xrightarrow{k_3} S_1 K \xrightarrow{k_4} \ldots \\
	\ldots S_2 F \xrightarrow{k_3} S_1 F \xrightarrow{k_4} \ldots	
\end{gather*}

It was shown that processive systems are globally stable and distributive ones may be bistable (having multiple steady states).

There are also extra classifications in double phosphorylation systems:

The order phosphorylation occurs matters too, so if:
\rom{1}: the \text{last} phosphorylated site is being dephosphorylated in the first dephosphorylation binding, then the mechanism is said to be \textbf{sequential}.

\rom{2} Whereas, if the \text{first} phosphorylated site is being \textbf{de}phosphorylated first then the system is \textbf{cyclic}.

You can see here an example of this classification in two mechanisms that, while they differ in this category, they are still both distributive.\footcite{conradi2024}
\begin{figure}[H]	
	\includegraphics[width=13cm]{math_pics/cyclic-vs-sequential.png}\label{cyclic_vs_sequential_figure}
	\centering
\end{figure}
In the left one here, $S_{ij}$ shows which one of the $2$ binding sites gets phosphate groups attached to it, (Ex, $S_{10}$: first one is phosphorylated, second not).

We'll now try to solve the problem whether a particular system undergoes simple Hopf bifurcations.

Take, for example, the following basic cycle:

\textbf{Example 1} $(\mathcal{N}_1)$:
\begin{figure}[H]
	\centering
	\includegraphics[width=38mm]{math_pics/ex1-no-bifurcations.png}	
	\caption{Network $( \mathcal{N}_1 )$ \footcite{doi:10.1098/rsif.2014.1405}}
\end{figure}

Whose chemical reaction notation looks like:
\begin{gather}\label{network1}
	K + S_0 \xrightleftharpoons[k_2]{k_1} K S_0 \xrightarrow{k_3} K + S_1 \tag{$\mathcal{N}_1$} \\
	F + S_1 \xrightleftharpoons[k_4]{k_5} F S_1 \xrightarrow{k_6} F + S_0 \notag
\end{gather}

\textbf{Remark 1:} $(a)$ As you can see, the network (\ref{network1}) consists of $6$ species ($2$ substrates $S_0$, $S_1$; $2$ enzymes $K$, $F$; which together form $2$ complexes: $K S_0$, $F S_1$) and $6$ reactions ($2$ reversible and $2$ irreversible) with matrix subspace $=$ rank $\Gamma = 3$.

Using \href{https://github.com/viktorashi/Open-CoNtRol}{the app}, talked more about in Chap.\ref{ch:web-app}, I've generated the following stoichiometric matrices, its rank and its corresponding dynamical system:

\begin{align*}
	\Gamma &=
	\begin{pmatrix}
		K & -1 &  1 &  1 &  0 &  0 &  0 \\
		S0& -1 &  1 &  0 &  0 &  0 &  1 \\
		KS0 &  1& -1& -1 &  0 &  0 &  0 \\
		S1 &  0 &  0 &  1& -1 &  1 &  0 \\
		F &  0 &  0 &  0& -1 &  1 &  1 \\
		FS1 &  0 &  0 &  0 &  1& -1& -1  \\		
	\end{pmatrix}, \quad \text{rank}(\Gamma) = 3 \\[3ex]
	\Gamma_L &=
	\begin{bmatrix}
		1&0&0&0&0&0\\
		1&0&0&0&0&0\\
		0&1&1&0&0&0\\
		0&0&0&1&0&0\\
		0&0&0&1&0&0\\
		0&0&0&0&1&1
	\end{bmatrix} \\[3ex]
	&
	\begin{cases*}
		\begin{array}{ll}
			\dot{x}_1(t) = -k_4 x_1(t) *x_6(t)+k_5 x_2(t)+k_6 x_2(t) \\
			\dot{x}_2(t) = k_4 x_1(t) *x_6(t)-k_5 x_2(t)-k_6 x_2(t) \\
			\dot{x}_3(t) = -k_1 x_3(t) *x_5(t)+k_2 x_4(t)+k_3 x_4(t) \\
			\dot{x}_4(t) = k_1 x_3(t) *x_5(t)-k_2 x_4(t)-k_3 x_4(t) \\
			\dot{x}_5(t) = -k_1 x_3(t) *x_5(t)+k_2 x_4(t)+k_6 x_2(t) \\
			\dot{x}_6(t) = k_3 x_4(t)-k_4 x_1(t) *x_6(t)+k_5 x_2(t) \\
		\end{array}	
	\end{cases*}
\end{align*}
Where the "species to index mapping" is:
\[
	F:  x_1
 | FS1: x_2
 | K: x_3
 | KS0: x_4
 | S0: x_5
 | S1: x_6
\]
\textbf{Remark 1:} $(b)$ For the network (\ref{network1}), the cone (defined as in \ref{flux_cone}) $\Gamma v(k,x) = 0, \forall v \geqq 0$ is spanned by the vectors $w_0, w_1, w_2 \geqq 0 : \lambda_0 w_0 + \lambda_1 w_1 + \lambda_2 w_2 \geqq 0 \iff \lambda_1, \lambda_2, \lambda_3 \geq 0$.

Now we may consider the Jacobian written in convex parameters for network (\ref{network1}), $J_1(h,\lambda)$ as defined in \ref{jacobian_convex_params}, which is parametrized by $9$ parameters: $\lambda_0, \lambda_1, \lambda_2$ and $h_1, \ldots, h_6$.

So, from \textbf{Remark 1} $(a)$ and $(b)$, it follows that the characteristic polynomial of $J_1(h, \lambda)$ is
\[
	p_{h,\lambda}(z) = z^3 (a_0(h,\lambda) z^3 + a_1(h, \lambda)z^2 + a_2 (h,\lambda)z + a_3(h,\lambda) )
\]
Where each coefficient $a_i, i = \overline{0,3}$ depends on these $9$ parameters.

Now, with the motivation of (\ref{ruling_out_simple_hopf_bif}) in mind, we compute the values of $a_i(h,\lambda), i = \overline{0,3}$ and $\det H_i(h,\lambda), i = \overline{1,2}$ according to (\ref{hurwitz_theorem}), obtaining the following proposition:
\begin{proposition}
	Based on the notation above, for the network (\ref{network1}):

	$\det H_1(h,\lambda)$, $\det H_2(h,\lambda)$ and $a_3(h,\lambda)$ contain only positive monomials. Thus, $\det H_1(h,\lambda)$, $\det H_2(h,\lambda) \geq 0, \forall h, \lambda \geqq 0$, specifically $\det H_2(h,\lambda) \neq 0$, $\lightning$ (\ref{ruling_out_simple_hopf_bif}).
\end{proposition}
\begin{proof}
	Now to solve these we're going to use some Maple scripts that rely on the \cite{franz2016ConvexMaple} package (too large to include inside this document).

	Given the network structure defined above, as well as the $E\lambda$ matrix made of the extreme cone vectors (\textbf{ray}):
	\[
		E\lambda =
		\begin{bmatrix}
			\lambda_2 + \lambda_3 \\
			\lambda_1 \\
			\lambda_3 \\
			\lambda_2 + \lambda_3 \\
			\lambda_2 \\
			\lambda_3
		\end{bmatrix}
	\]
	The Jacobian in convex parameters, as given in (\ref{jacobian_convex_params}), using that script is:
	\[
		\left. J_1(h,\lambda)=
		\left[
			\begin{array}{ccccccc}-\lambda_1hl&-\lambda_1h2&0&0&0&0&\lambda_1h7\\-\lambda_1hl&-\lambda_1h2&\lambda_1h3&0&0&0&0\\\lambda_1hl&\lambda_1h2&-\lambda_1h3&0&0&0&0\\0&-\lambda_2h2&\lambda_1h3&(-\lambda_2-\lambda_1)h4&0&\lambda_2h6&0\\0&\lambda_2h2&0&\lambda_2h4&-\lambda_2h5&0&0\\0&0&0&0&\lambda_2h5&-\lambda_2h6&0\\0&0&0&\lambda_1h4&0&0&-\lambda_1h7
		\end{array}\right.\right]
	\]
	Then by means of another Maple script, we can write the Hurwitz matrices of its corresponding characteristic polynomial. (They are pretty huge and will always overflow \LaTeX)

	But just by looking at $H_2(h,\lambda)$, we can deduce that $a_{ij}$ is a positive monomial, for all  its entries $\forall h , \lambda > 0$, in particular $\det H_2(h, \lambda) \neq 0$.

	So, $\det H_1(k,x^*) > 0, \forall k, x^* > 0$, moreover the coefficient of the largest degree term in the characteristic polynomial $a_3(k, x^*) > 0$, so according to \ref{ruling_out_simple_hopf_bif} : For network (\ref{network1}), $\nexists k, x^* \geqq 0$ for which $J_1(k,x^*)$ has a pair of purely imaginary eigenvalues $\implies \nexists$ a simple Hopf bifurcation in our system for any rate constants or positive steady states
\end{proof}

\textbf{Example 2}:
Here we'll show how to prove the \textbf{existence} of a Hopf bifurcation in a network, which requires quite a bit of a lot of work.\footcite{conradi2024, Suwanmajo2020, Conradi2020} Let us take the cyclic example, Network $(a)$ from \ref{cyclic_vs_sequential_figure}, from which we can actually infer a particular case of networks with some nice properties.

We'll first need to define some continuations of convex parameters \ref{convex_params_definition}, aided by \ref{hurwitz_theorem} in regard to this particular case.

\hfill\break
\subsection{The Jacobian of networks with a single extreme vector}\label{jacobian_single_extreme_vector}

If (\text{no. of extreme rays}) $l = 1 \implies$ the cone ker$(\Gamma) \bigcap \mathbb{R}^r_{\geq 0}$ is spanned by a single extreme vector, so the Jacobian can be written more simply in terms of just it, and its corresponding scalar $\lambda$, in convex coordinates.
\begin{equation}\label{jacobian_convex_single_extreme_ray}
	J_\lambda(h)=\lambda \Gamma \operatorname{diag}(E)\Gamma_L^T\operatorname{diag}(h).
\end{equation}
Where its char. poly. is:
\begin{equation}\label{char_poly_j_lambda}
	\det(\mu I-J_\lambda(h))=\mu^{n-s}(\mu^s+a_1(\lambda,h)\mu^{s-1}+\ldots+a_s(\lambda,h))
\end{equation}
With $s := \text{rank}(J_\lambda) \leq n$.
But for the particular case $\lambda = 1$:
\begin{equation}\label{char_poly_j_1}
	\det(\mu I-J_1(h))=\mu^{n-s}(\mu^s+b_1(h)\mu^{s-1}+\ldots+b_s(h))
\end{equation}
Now we'll use a Corollary (the consequence of a Lemma discussed in another more in-depth section) to relate the general $\lambda > 0$ case with the $\lambda = 1$ case.
\begin{corollary}
	Let's say $E \in \mathbb{R}^r_{\geq 0}$. Let $J_\lambda(h)$ as in \ref{jacobian_convex_single_extreme_ray} with

	rank$(J_{\lambda}(h))=\mathrm{rank}(J_{1}(h))=s<n$ and $a_i(\lambda,h), b_i(h)$ coeff. of \ref{char_poly_j_lambda} and \ref{char_poly_j_1}. $\implies$
	\begin{equation}\label{colorally_jacobians_1st_satisfaction}
		a_i(\lambda,h)=\lambda^ib_i(h),i=1,\ldots,s.
	\end{equation}
	aaand
	\begin{equation}\label{colorally_jacobians_2nd_satisfaction}
		\begin{aligned}\det(\mu I_n-J_\lambda(h))&=\det(\mu I_n-\lambda J_1(h))\\&=\mu^{n-s}\lambda^s\left(\left(\frac{\mu}{\lambda}\right)^s+\sum_{i=1}^sb_i(h)\left(\frac{\mu}{\lambda}\right)^{s-i}\right)
		\end{aligned}
	\end{equation}
\end{corollary}
Sooo: \textbf{Remark: }

\rom{1} From \ref{colorally_jacobians_2nd_satisfaction} $\implies$ for $\omega(h) \in \text{Eig}(J_1(h)) \implies \mu(\lambda, \omega) = \lambda \omega(h) \in \text{Eig}(J_\lambda)(h)$.

\rom{2} $\mathrm{sign}(Re(\mu(\lambda,h)))=\mathrm{sign}(Re(\omega(h)))$

\rom{3} $J_\lambda(h)$ has an $\mu \in$Eig with Re$(\mu) = 0$ of the form $\pm i \lambda w(h) \iff J_1$ has one of the form $\pm i \omega (h)$.

Now, utilizing this result, along with \ref{yangs_criterion}, we may find sufficient conditions for bifurcations in single ray-defined networks

Constructing Hurwitz determinants:
\begin{gather*}
	H_l(\lambda, h) \text{of \ref{colorally_jacobians_1st_satisfaction}} \\
	\text{and} \\
	G_l(h) \text{of \ref{colorally_jacobians_2nd_satisfaction}}
\end{gather*}

\begin{proposition}
	We get the following relationship:
	\[
		\det\left(H_l(\lambda,h)\right)=\lambda^{l(l+1)/2}\det\left(G_l(h)\right),l=\overline{1,s}
	\]
	So we basically only need \ref{colorally_jacobians_2nd_satisfaction} and to study the $G_i(h), i = \overline{1,s}$ Hurwitz matrices if the system's convex polyhedral cone ker$\Gamma \bigcap \mathbb{R}^r_{\geq 0}$.
\end{proposition}
\begin{proposition}
	For $\dot{x} = \Gamma v(k,x)$, ODE system for CRN with mass-action kinetics, rank$(\Gamma) = s$. Suppose the Matrix E from \ref{convex_params_definition} is a single extreme ray, and the Jacobian $J_\lambda(h)$ is as in \ref{jacobian_convex_single_extreme_ray}.
	Also, let the char. polynomials of $J_\lambda(h)$ and $J_1(h)$ be as in \ref{char_poly_j_lambda} and \ref{char_poly_j_1}, respectively.

	If $\exists h = h^* :$
	\[
		\begin{aligned}
			b_s(h^*)&\mathrm{>0~and}\\\det(G_1(h^*))&>0,...,\det(G_{s-2}(h^*))>0\mathrm{and}\\\det(G_{s-1}(h^*))&=0,
		\end{aligned}
	\]
	then:

	\rom{1} $J_1(h^*)$ has a single pair of purely imaginary eigenvalues $\mu = \pm i \omega(h^*)$.

	\rom{2} $J_\lambda(h^*)$ has a single pair of purely imaginary eigenvalues $\mu = \pm i \lambda \omega(h^*), \forall \lambda > 0$.

	\rom{3} for $\dot{x} = \Gamma v(k,x), \exists$ simple Hopf	 bifurcation at $h = h^*$, $\forall \lambda > 0$ if $\exists i \in \left\{ 1 , \ldots , n \right\}:$
	\[
		\frac{\partial\det(G_{n-1})}{\partial h_i}|_{h_i=h_i^*}\neq0.
	\]
	Meaning, it crosses the axis at that point.
\end{proposition}
Hence, because $\det(H_{s-1}(h,\lambda))=\lambda^{\frac{s(s-1)}{2}}\det(G_{s-1}(h))$, $\lambda$ is not a well-suited bifurcation parameter, since $\frac{\partial\det(H_{s-1}(h,\lambda))}{\partial\lambda}|_{h=h^{*}}=0$ when $\det(G_{s-1}(h^{*}))=0$. It all basically comes down to $h$, not $\lambda$.
\hfill\break

Now, let's get into the bread of it. We may re-define the cyclic and distributive system we're going to discuss again.
\begin{figure}[H]
	\includegraphics[width=13cm]{math_pics/cyclic-distributive-n1.png}	
	\centering
	\caption{$\mathcal{N}_2$ \footcite{conradi2024}}
\end{figure}
Defined by these reactions:
\begin{align}\label{network2}
	S_{00} + K \xrightleftharpoons[ \kappa_2 ]{\kappa_1} KS_{00} \xrightarrow{\kappa_3} S_{10} + K \xrightleftharpoons[ \kappa_5 ]{\kappa_4} KS_{10} \xrightarrow{\kappa_6} S_{11} + K \tag{$\mathcal{N}_2$}
	\\
	S_{11} + F \xrightleftharpoons[ \kappa_8 ]{\kappa_7} FS_{11} \xrightarrow{\kappa_9} S_{01} + F \xrightleftharpoons[ \kappa_{11} ]{\kappa_{10}} FS_{01} \xrightarrow{\kappa_{12}} S_{00} + F \ \notag
\end{align}
We'll look into just the irreversible sub-network, though.
\begin{align}\label{network2_irr}
	S_{00} + K \xrightarrow{\kappa_1} KS_{00} \xrightarrow{\kappa_3} S_{10} + K \xrightarrow{\kappa_4} KS_{10} \xrightarrow{\kappa_6} S_{11} + K \tag{$\mathcal{N}_2 / IR$}
	\\
	S_{11} + F \xrightarrow{\kappa_7} FS_{11} \xrightarrow{\kappa_9} S_{01} + F \xrightarrow{\kappa_{10}} FS_{01} \xrightarrow{\kappa_{12}} S_{00} + F \ \notag
\end{align}
And using the following index-to-species mapping:
\[
	\begin{array}{ccccccccccc}x_{1}&x_{2}&x_{3}&x_{4}&x_{5}&x_{6}&x_{7}&x_{8}&x_{9}&x_{10}\\K&F&S_{00}&S_{10}&S_{01}&S_{11}&KS_{00}&KS_{10}&FS_{01}&FS_{11}
	\end{array}
\]
We may obtain their system of ODEs:

$\mathcal{N}_2$:
\[
	\begin{aligned}
		\dot{x}_1&=-\kappa_{1}x_{1}x_{3}-\kappa_{4}x_{1}x_{4}+(\kappa_{2}+\kappa_{3})x_{7}+(\kappa_{5}+\kappa_{6})x_{8}\\\dot{x}_2&=-\kappa_{10}x_{2}x_{5}-\kappa_{7}x_{2}x_{6}+(\kappa_{8}+\kappa_{9})x_{10}+(\kappa_{11}+\kappa_{12})x_{9}\\\dot{x}_3&=-\kappa_1x_1x_3+\kappa_2x_7+\kappa_{12}x_9\\\dot{x}_4&=-\kappa_4x_1x_4+\kappa_3x_7+\kappa_5x_8\\\dot{x}_5&=-\kappa_{10}x_2x_5+\kappa_9x_{10}+\kappa_{11}x_9\\\dot{x}_6&=-\kappa_7x_2x_6+\kappa_8x_{10}+\kappa_6x_8\\\dot{x}_7&=-(\kappa_2+\kappa_3)x_7+\kappa_1x_1x_3\\\dot{x}_8&=-(\kappa_5+\kappa_6)x_8+\kappa_4x_1x_4\\\dot{x}_9&=-(\kappa_{11}+\kappa_{12})x_9+\kappa_{10}x_2x_5\\\dot{x}_{10}&=-(\kappa_8+\kappa_9)x_{10}+\kappa_7x_2x_6
	\end{aligned}
\]
$\mathcal{N}_2 / IR$:
\[
	\begin{aligned}
		\dot{x}_{1} &= -\kappa_{1} x_{1} x_{3} - \kappa_{4} x_{1} x_{4} + \kappa_{3} x_{7} + \kappa_{6} x_{8},\\
		\dot{x}_{2} &= \kappa_{9} x_{10} - \kappa_{10} x_{2} x_{5} - \kappa_{7} x_{2} x_{6} + \kappa_{12} x_{9},\\
		\dot{x}_{3} &= -\kappa_{1} x_{1} x_{3} + \kappa_{12} x_{9},\\
		\dot{x}_{4} &= -\kappa_{4} x_{1} x_{4} + \kappa_{3} x_{7},\\
		\dot{x}_{5} &= \kappa_{9} x_{10} - \kappa_{10} x_{2} x_{5},\\
		\dot{x}_{6} &= -\kappa_{7} x_{2} x_{6} + \kappa_{6} x_{8},\\
		\dot{x}_{7} &= \kappa_{1} x_{1} x_{3} - \kappa_{3} x_{7},\\
		\dot{x}_{8} &= \kappa_{4} x_{1} x_{4} - \kappa_{6} x_{8},\\
		\dot{x}_{9} &= \kappa_{10} x_{2} x_{5} - \kappa_{12} x_{9},\\
		\dot{x}_{10} &= -\kappa_{9} x_{10} + \kappa_{7} x_{2} x_{6}.
	\end{aligned}
\]
Now there are 3 conservation laws for both networks, diminishing the rank of the Jacobian matrix to $7$. These are:
\[
	\begin{aligned}
		x_{8}+x_{7}+x_{1} &=c_{1}\\
		x_{10}+x_{9}+x_{2}&=c_{2}\\
		x_6 + x_3 + x_{ 10 } + x_5 + x_7 + x_4 + x_9 + x_8 &= c_3
	\end{aligned}
\]
$E \in \mathbb{R}^8$ extreme ray matrix of $\mathcal{N}_3$ is:
\begin{equation}\label{e_matix_net3}
	E^T=(1,1,1,1,1,1,1,1)
\end{equation}
Hence, we can use \ref{jacobian_convex_single_extreme_ray}. So with $E$ of \ref{e_matix_net3} and \ref{convex_params_definition}, which can be written as:
\begin{gather*}
	\kappa_1x_1x_3=\kappa_3x_7=\kappa_4x_1x_4=\kappa_6x_8=\kappa_7x_2x_6=\kappa_9x_{10}=\kappa_{10}x_2x_5=\kappa_{12}x_9=\lambda. \\
	\Downarrow \\
	x_{3} = \frac{\lambda}{\kappa_{1} x_{1}}, \ x_{4} = \frac{\lambda}{\kappa_{4} x_{1}}, \ x_{5} = \frac{\lambda}{\kappa_{10} x_{2}}, \ x_{6} = \frac{\lambda}{\kappa_{7} x_{2}} \\
	x_{7} = \frac{\lambda}{\kappa_{3}}, \ x_{8} = \frac{\lambda}{\kappa_{6}}, \ x_{9} = \frac{\lambda}{\kappa_{12}}, \ x_{10} = \frac{\lambda}{\kappa_{9}} \\
	\forall x_1, x_2 > 0 \\
	\Downarrow \\
	\kappa_{1} = h_{1} h_{3} \lambda, \kappa_{3} = h_{7} \lambda, \kappa_{4} = h_{1} h_{4} \lambda, \kappa_{6} = h_{8} \lambda \\
	\kappa_{7} = h_{2} h_{6} \lambda, \kappa_{9} = h_{10} \lambda, \kappa_{10} = h_{2} h_{5} \lambda, \kappa_{12} = h_{9} \lambda \\
	\text{with } h_i = \frac{1}{x_i} , i = \overline{1,10}
\end{gather*}
Now using \cite{franz2016ConvexMaple} for $J_\lambda(h)$ of \ref{network2_irr} as defined in \ref{jacobian_convex_single_extreme_ray} is:
\begin{align}
	J_\lambda(h) = \lambda
	\left.\left[
		\begin{array}{rrrrrrrrrr}-2h_{1}&0&-h_{3}&-h_{4}&0&0&h_{7}&h_{3}&0&0\\0&-2h_{2}&0&0&-h_{5}&-h_{6}&0&0&h_{9}&h_{10}\\-h_{1}&0&-h_{3}&0&0&0&0&0&h_{3}&0\\-h_{1}&0&0&-h_{4}&0&0&h_{7}&0&0&0\\0&-h_{2}&0&0&-h_{5}&0&0&0&0&h_{10}\\0&-h_{2}&0&0&0&-h_{6}&0&h_{3}&0&0\\h_{1}&0&h_{3}&0&0&0&-h_{7}&0&0&0\\h_{1}&0&0&h_{4}&0&0&0&-h_{3}&0&0\\0&h_{2}&0&0&h_{5}&0&0&0&-h_{9}&0\\0&h_{2}&0&0&0&0&h_{6}&0&0&-h_{10}
	\end{array}\right.\right]
\end{align}
with $rank(J_\lambda(h)) = 7$ and characteristic polynomial is of the form:
\[
	\det\left(\mu I-J_\lambda(h)\right)=\mu^3\left(\mu^7+\lambda b_1(h)\mu^6+\ldots+\lambda^6b_6(h)\mu+\lambda^7b_7(h)\right),
\]
With $b_i(h)$ from char. poly of $J_1(h)$.
Now, constructing the Hurwitz matrices $G_i(h), i = \overline{1,6}$. Assume that at a fixed $h = h^*$ the following occur.
\begin{equation}\label{bifurcation_main_condition}
	\begin{aligned}
		b_{7}(h^{*})&>0 \text{ and } \\
		\det(G_1(h^*))&>0,\ldots,\det(G_5(h^*))>0 \text{ and }\\
		\det(G_6(h^*))&=0.
	\end{aligned}
\end{equation}
Then, if additionally:
\begin{equation}\label{bifurcation_transversality_condition}
	\exists l\in\{1,\ldots,10\} : \frac{\partial\det(G_6)}{\partial h_l}|_{h=h^*}\neq0\quad.
\end{equation}

\ref{network2_irr} has a simple Hopf bifurcation at $h = h^*, \forall \lambda > 0$.

Both $\left\{ b_0(h), \ldots, b_5(h) \right\} \bigcup \left\{   b_7(h) \right\}$ and
$\det(G_i(h)), i = \overline{2,5}$ are made up of only positive monomials, whereas $\det(G_6(h))$ has terms with both signs, so it may have roots.

In the following section we'll talk about $\det(G_6(h))$'s potential to have roots.

First (and I swear this is relevant), let's look at some shapes:

These are all (k)-\textbf{polytopes}, shapes whose faces are flat (k-1)-\textbf{polytopes}.
\begin{figure}[H]
	\centering
	\includegraphics[width=13cm]{math_pics/polytopes.png}
	\caption{Polytopes \cite{Brandenburg2024}}
\end{figure}
Now, if their vertices are exactly at the integer Cartesian coordinates, then they are called \textbf{integer polytopes}, meaning the polytope itself is the \textbf{convex hull} of the integer points inside it.

Now we can use this to asymptotically analyze multivariate polynomials, by looking at the \textbf{Newton polytope} of the polynomial, which is the convex hull of the integer points in $\mathbb{R}^n$ corresponding to the monomials of the polynomial.

Taking a vector $y = (y_1, \ldots, y_n)$, and $(a_k)_k \subseteq \mathbb{N}^n$ pair-wise distinct, consider
\[
 f(y) = \sum_{k} c_k y^{a_k}
\]
where $(x_1, \ldots, x_n)^{(y_1, \ldots, y_n)}$ is a representation of the monomial $x_1^{y_1} x_2^{y_2} \ldots x_n^{y_n}$.
Now, the \textbf{Newton polytope} of that polynomial is.
\[
	\mathrm{Newt}(f)=\left\{\sum_k\alpha_k\mathbf{a}_k \middle|\sum_k\alpha_k=1\And\forall j\alpha_j\geq0\right\}, \forall c_k \neq 0_n
\]
This structure offers a lot in insight into where (or if) roots might be.
Take now, the Newton Polytope of $\det G_6(h)$, which by using software like \href{https://polymake.org/doku.php/start}{Polymake}\footcite{Assarf2017, Gawrilow2000} we can obtain its hyper-plane intersection form
% create matrix with no borders like no parantheses, just like a table
\begin{align*}
	\left.
	\begin{array}{ccc}
		-h_{1} \geq -6 &
		-h_{1} - h_{2} \geq -11 &
		-h_{1} - h_{2} - h_{3} \geq -15 \\
		-h_{1} - h_{2} - h_{3} - h_{6} \geq -18 &
		h_{6} \geq 0 &
		-h_{2} \geq -6 \\
		-h_{1} - h_{3} \geq -11 &
		-h_{3} - h_{6} \geq -11 &
		-h_{2} - h_{3} - h_{6} \geq -15 \\
		-h_{1} - h_{3} - h_{6} \geq -15 &
		h_{1} \geq 0 &
		-h_{2} - h_{3} \geq -11 \\
		-h_{2} - h_{6} \geq -11 &
		-h_{1} - h_{2} - h_{6} \geq -15 &
		-h_{3} \geq -6 \\
		h_{2} \geq 0 &
		-h_{1} - h_{6} \geq -11 &
		-h_{6} \geq -6 \\
		h_{3} \geq 0
	\end{array} \right.
\end{align*}
By treating $G_6(h)$ like an $h_1, h_2, h_3 \text{ and }, h_6$ polynomial, we observe the factor
\[
	h_{10}h_7 - h_8 h_9
\]
pops out. Now we might want to make the monomials whose factors are that, dominant, by performing the restriction / transformation;
\begin{equation}\label{h_substitution}
	h_1\to t,h_2\to t,h_3\to t,h_6\to t.
\end{equation}
Now we represent $\det G_6(h)$ as a polynomial of variable $t$ with coefficients in all other h's:
\begin{equation}\label{6th_hurwitz_det}
	\begin{aligned}
		\det G_6(t)
		&=324t^{18}(h_{10}+h_7)(h_{10}h_7-h_8h_9)+\ldots\\&+h_{10}h_4h_5h_7h_8h_9\\&\cdot(h_{10}+h_4)(h_{10}+h_5)(h_{10}+h_7)(h_{10}+h_8)(h_{10}+h_9)\\&\cdot(h_4+h_5)(h_4+h_7)(h_4+h_8)(h_4+h_9)\\&\cdot(h_5+h_7)(h_5+h_8)(h_5+h_9)\\&\cdot(h_7+h_8)(h_7+h_9)\\&\cdot(h_8+h_9).
	\end{aligned}	
\end{equation}
Since the only $minuses$ that might pollute its positivity are in the "$324 t^{18}$" term $\implies \det G_6(0) > 0$, so if
\begin{gather}\label{bif_ineq_condition}
	h_{10}h_{7} - h_{8}h_{9} < 0 \\
	\Downarrow \notag \\
	\exists t_1 > 0 : \det G_6(t_1) = 0, \det G_6(t) < 0, \forall t > t_1 \notag
\end{gather}
by the Intermediate Value Theorem, meaning it passes $0$.

So it means we've shown a root exists for $\det G_6(h)$, and hence a Hopf bifurcation may occur in the system.

Where though? This is where the second part of the problem comes in.

\begin{itemize}
	\item \rom{1} We may experimentally just pick and choose values that verify	\ref{bifurcation_main_condition} and \ref{bif_ineq_condition}, for exmaple, \cite{conradi2024} did $ h_9 = 2, h_8 = h_7 = h_{ 10 } = 1$
	\item \rom{2} $\forall h_4, h_5 > 0$, like $h_5 = h_4 = 1$
	\item \rom{3} Now taking \ref{h_substitution} along with the previous values and using them in \ref{6th_hurwitz_det} we get
		\begin{gather*}
			497664+10946304t+103721056t^{2}+579850652t^{3}+2169242876t^{4} \\
			+5787611019t^{5}+113986711182t^{6}+16865933820t^{7} \\
			+18863357157t^{8}+15900121640t^{9}+9989687485t^{10} \\
			+4589099030t^{11}+1497364081t^{12}+331280824t^{13} \\
			+45135703t^{14}+2794428t^{15}-85122t^{16}-20304t^{17}-648t^{18}=0	
		\end{gather*}
	\item \rom{4} Now taking its root $\implies t^* \approx 14.8$
	\item \rom{5} Using these to verify \ref{bifurcation_transversality_condition}:
		\begin{align*}
			h^* =
			\begin{pmatrix}
				14.8 \\
				14.8 \\
				14.8 \\
				1 \\
				1 \\
				14.8 \\
				1 \\
				1 \\
				2 \\
				1
			\end{pmatrix}
		\end{align*}
	\item \rom{6} Taking $t > t^*$, say $15$, and solving for \ref{k_convert_back_from_convex} and $x = \frac{1}{h}$:
		\begin{align}
			h &= \left(15,15,15, 1,1,15,1,1,2,1\right)^T \\
			x &= \left( \frac{1}{15},\frac{1}{15},\frac{1}{15},1,1,\frac{1}{15},1,1,\frac{1}{2},1 \right)^T \\
			k &= \left( 225 \lambda, \lambda, 15\lambda, 225\lambda, \lambda, 15\lambda, 2\lambda  \right) \\
			&= (t^{2},1,t,1,t^{2},1,t,2)
		\end{align}
\end{itemize}
Considering $\lambda = 1$ and putting these into \href{https://github.com/viktorashi/Open-CoNtRol}{the app} we get something that looks like:
\begin{figure}[H]
	\includegraphics[width=13cm]{math_pics/oscillations.png}
	\centering
	\label{fig:x3_x6_osccilations}
	\caption{Oscillations of $S00, S11$}
\end{figure}

They're oscillating! We've found a bifurcation point!

But wait, in the beginning we reduced our system from \ref{network2} to \ref{network2_irr}, taking the reversible reactions with us. But luck would have it that \cite{banaji2017} shows that adding a reaction who's reaction vector is a linear combination of the other reactions, and hence preserves rank$\Gamma$ does, in fact mean the resulting system preserves its capacity for bifurcations.

So adding back
\begin{gather*}
	KS_{00} \xrightarrow{k_2} K \\
	KS_{10} \xrightarrow{k_5} S_{10} + K \\
	FS_{11} \xrightarrow{k_8} S_{11} + F \\
	FS_{01} \xrightarrow{k_{11}} S_{01} + F
\end{gather*}
Into our system, we obtain bifurcation parameters and initial conditions for which oscillations occur, namely:
\[
	x =
	\begin{bmatrix*}
		0.0885267\\
		0.0528367\\
		0.0496013\\
		0.74587\\
		1.261\\
		0.084892\\
		0.97124\\
		1.0069\\
		0.49383\\
		1.02
	\end{bmatrix*}
\]
and $k$ as before, but with the extra reaction rates being simply a constant we can vary to see how the system looks $k_b = k_5 = k_2 = k_{11} = k_8$. Thus we get:

\begin{figure}[H]
	\includegraphics[width=13cm]{math_pics/oscillations-full-network.png}
	\centering
	\label{fig:full_net_oscillations}
	\caption{Oscillations of the full \ref{network2} network}
\end{figure}

There we have, a "visual" proof, you could say.

\hfill\break
\hfill\break

\textbf{Example 3}:
Now let us apply the same intuition for the following cyclic network (\ref{network3}).

\begin{figure}[H]
	\includegraphics[width=13cm]{math_pics/ex3-bifurcations.png}
	\centering
\end{figure}

\hfill\break
//TODO: sa nu uiti ca am plotuit deja rezultatele de la networku providenind din asta, simplificat in \ref{ch:web-app}
\hfill\break

Its chemical reaction network is:
\begin{align}\label{network3}
	S_{0} + K \xrightleftharpoons[ \kappa_2 ]{\kappa_1} KS_{0} \xrightarrow{\kappa_3} K S_1 \xrightleftharpoons{\kappa_4} S_2 + K \tag{$\mathcal{N}_3$} \\
	S_{2} + F \xrightleftharpoons[ \kappa_6 ]{\kappa_5} FS_{2} \xrightarrow{\kappa_7} S_1 + F \xrightleftharpoons[\kappa_9]{\kappa_8} FS_1 \xrightarrow{\kappa_{10}} S_0 + F \notag
\end{align}

By the same results used in \ref{network2}, we go $1$ level up the simplification stack, removing the reversible reactions and re-naming the reaction constants $\kappa_i$ like so:
\begin{align}\label{network3_irr}
	S_{0} + K \xrightarrow{\kappa_1} KS_{0} \xrightarrow{\kappa_2} K S_1 \xrightarrow{\kappa_3} S_2 + K \tag{$\mathcal{N}_3 / IR $} \\
	S_{2} + F \xrightarrow{\kappa_4} FS_{2} \xrightarrow{\kappa_5} S_1 + F \xrightarrow{\kappa_6} FS_1 \xrightarrow{\kappa_{7}} S_0 + F \notag
\end{align}

By the results used previously, we find that if this network is to have bifurcations, then adding the inverse reactions back would mean \ref{network3} also does. Going up $1$ more level, using \textbf{Theorem 3.2} from \cite{banaji2017} we see that we can

\chapter{CoNtRoL Simulations Web Application}
\label{conclusions}

\par
\textit{ Here we will present the open-source Web application developed for easing work involving chemical reaction networks, hopefully for students and researchers one day. It can be used for obtaining numerical analysis of chemical reaction networks, as well as plotting species against time, one another and so on. We'll present the technologies used throughout the project, how to run it locally and a couple of usage examples involving what we've presented in the previous chapters. }

\subsection{Overview of the technologies}
The website is a back-end application built in \textbf{Python}, a programming language known for its use in basically every single science, including natural sciences so it's a no-brainer when in comes to plotting.
The web server is built using \textbf{Flask}, a lightweight web app framework and the webpages served are server-side rendered by Flask's template engine depedency - \textbf{Jinja}.\\
\\
The crux of the functionality is aided by the Python library \textbf{Tellurium}; which is, as their docs say; "A Python Environment for Reproducible Dynamical Modeling of Biological Networks". It uses a subset of the Systems Biology Markup Language ( \textbf{SBML}) called \textbf{Antimony} which can be used in this app to create a Chemical Reaction Network, as well as the friendlier selects form. So the bits doing the magic are the calls to \verb|road_runner.loada()| which are used to \textbf{\textit{load}}    \textbf{a}ntimony code into the model.
the \verb|road_runner..simulate()| function is then used for running and obtaining sumulation data, followed by \verb|road_runner.plot()|, which in turn calls a \verb|matplotlib| headless backend for writing the plotted results to a file.

\subsection{Running it}
As explained in the \verb|README| of \href{https://github.com/viktorashi/Open-CoNtRol}{the project}, running locally is done automatically by the \verb|run_script.sh|, which figures out your machine's local IP, sets the required environment variables and runs \\
\verb|flask --debug run --host="$ip"|. Output regarding traffic to the server as well as internal workings of the app will now be redirected to \verb|stdout| of the terminal. \\\\\\\\\\
All the user has to do beforehand is:

\tikz\draw[black,fill=black] (0,0) circle (.5ex); clone the project

\begin{lstlisting}[language=bash]
    git clone https://github.com/viktorashi/Open-CoNtRol.git
\end{lstlisting}

\tikz\draw[black,fill=black] (0,0) circle (.5ex); \textbf{c}hange \textbf{d}irectory into it.

\begin{lstlisting}[language=bash]
    cd Open-CoNtRol
\end{lstlisting}

\tikz\draw[black,fill=black] (0,0) circle (.5ex); install the requirements

\begin{lstlisting}[language=bash]
    pip install -r requirements.txt
\end{lstlisting}

\tikz\draw[black,fill=black] (0,0) circle (.5ex);
give the \verb|run_script.sh| execute permissions

\begin{lstlisting}[language=bash]
    chmod +x ./run_script.sh
\end{lstlisting}

\tikz\draw[black,fill=black] (0,0) circle (.5ex);
and finally run it

\begin{lstlisting}[language=bash]
    ./run_script.sh
\end{lstlisting}

Among the wall of output will also be the line showing the address your server is located at, for example:
\verb|* Running on http://192.168.0.94:5000|, address at which you'll be greeted with this screen

\[
    \includegraphics[width=13cm]{app_photos/control_home_screen.png}\\
\]

You can either use this as a starting point or the page located at the \verb|/antimony| path:

\[
    \includegraphics[width=13cm]{app_photos/control_antimony_home.png}\\
\]
Both of these redirect to \verb|/numerical_analysis| analysing the system \ldots well, numerically.
As an example, this Antimony code:

\begin{lstlisting}
S0 -> KS1; k1*S0
KS1 -> S2; k2*KS1
S2 + F -> FS2; k3*S2*F
FS2 -> F; k4*FS2
F -> S0 + F; k5*F
\end{lstlisting}

and its longer to write alternative:

\[
    \includegraphics[width=13cm]{app_photos/creca-prima-data-cand-am-scris-asta.png}\\
\]

both yield the same numerical analysis results:

\[
    \includegraphics[width=13cm]{app_photos/numerical_analysis.png}\\
\]

from here, one can choose from a selection of graphs they can represent given this system, the default one being the time series representation:

\[
    \includegraphics[width=13cm]{app_photos/tsr-input.png}\\
\]

from which we fill out the initial values of the concentrations for each species as, well as the reaction rates. So given, for example the values:

\begin{lstlisting}
    F = 0.874108
    FS2 = 7.620157734
    KS1 = 7.620157734
    S0 = 7.270157734
    S2 = 0.6000000000

    k1 = 0.1329759342
    k2 = 0.1329759342
    k3 = 2
    k4 = 0.1329759342
    k5 = 1
\end{lstlisting}

outcomes the graph:

\[
    \includegraphics[width=13cm]{app_photos/a-graph-indeed.png}\\
\].

\subsection{The bottom line}

So this is how the workflow of the app typically goes. \\
$
    \begin{WithArrows} & \textit{write out your system} \rightarrow \textit{get numerical analysis} \rightarrow \textit{choose your desired graph} \Arrow{\textit{fill out data values}}  \\
        & \hfill \textbf{Voilà}
    \end{WithArrows}
$

%\addcontentsline{toc}{chapter}{Concluzii}
%\addcontentsline{toc}{chapter}{Conclusions}

\bibliography{references.bib}

\end{document}
